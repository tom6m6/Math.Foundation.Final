%-----------------------------------------------------主文档 格式定义---------------------------------
\addtolength{\headsep}{-0.1cm}        %页眉位置
%\addtolength{\footskip}{0.4cm}       %页脚位置
%-----------------------------------------------------设定字体等------------------------------
\setmainfont{Times New Roman}    % 缺省字体
%\setCJKfamilyfont{song}{SimSun}
%\setCJKfamilyfont{hei}{SimHei}
%\setCJKfamilyfont{kai}{KaiTi}
%\setCJKfamilyfont{fs}{FangSong}
%\setCJKfamilyfont{li}{LiSu}
%\setCJKfamilyfont{you}{YouYuan}
%\setCJKfamilyfont{yahei}{Microsoft YaHei}
%\setCJKfamilyfont{xingkai}{STXingkai}
%\setCJKfamilyfont{xinwei}{STXinwei}
%\setCJKfamilyfont{fzyao}{FZYaoTi}
%\setCJKfamilyfont{fzshu}{FZShuTi}
%-------------------------------------------------------------------
\newCJKfontfamily\song{SimSun}
\newCJKfontfamily\hei{SimHei}
\newCJKfontfamily\kai{KaiTi}
\newCJKfontfamily\fs{FangSong}
\newCJKfontfamily\li{LiSu}
\newCJKfontfamily\you{YouYuan}
%\newCJKfontfamily\yahei{Microsoft YaHei}
\newCJKfontfamily\xingkai{STXingkai}
\newCJKfontfamily\xinwei{STXinwei}
\newCJKfontfamily\fzyao{FZYaoTi}
\newCJKfontfamily\fzshu{FZShuTi}
%------------------------------------------------------------------------------------------------------
\newcommand{\linghao}{\fontsize{48pt}{60pt}\selectfont}      % 零号, 1.4倍行距
\newcommand{\chuhao}{\fontsize{42pt}{\baselineskip}\selectfont}  % 初号& 42pt
\newcommand{\xiaochuhao}{\fontsize{36pt}{\baselineskip}\selectfont} % 小初号& 36pt
\newcommand{\yihao}{\fontsize{28pt}{38pt}\selectfont}        % 一号, 1.4倍行距
\newcommand{\erhao}{\fontsize{21pt}{28pt}\selectfont}        % 二号, 1.25倍行距
\newcommand{\xiaoer}{\fontsize{18pt}{18pt}\selectfont}       % 小二, 单倍行距
\newcommand{\sanhao}{\fontsize{16pt}{24pt}\selectfont}       % 三号, 1.5倍行距
\newcommand{\xiaosan}{\fontsize{15pt}{22pt}\selectfont}      % 小三, 1.5倍行距
\newcommand{\sihao}{\fontsize{14pt}{21pt}\selectfont}        % 四号, 1.5倍行距
\newcommand{\xiaosihao}{\fontsize{12pt}{18pt}\selectfont}    % 小四, 1.5倍行距
\newcommand{\dawuhao}{\fontsize{11pt}{11pt}\selectfont}      % 大五号, 单倍行距
\newcommand{\wuhao}{\fontsize{10.5pt}{10.5pt}\selectfont}    % 五号, 单倍行距
\newcommand{\xiaowuhao}{\fontsize{10pt}{10pt}\selectfont}    % 小五号, 单倍行距
\newcommand{\liuhao}{\fontsize{7.875pt}{7.875pt}\selectfont} % 六号, 单倍行距
\newcommand{\qihao}{\fontsize{5.25pt}{\baselineskip}\selectfont}
%--------------------------------------------------------------------------------------------------------
%-----------------------------------------------------------定义颜色---------------
\definecolor{blueblack}{cmyk}{0,0,0,0.35}%浅黑
\definecolor{darkblue}{cmyk}{1,0,0,0}%纯蓝
\definecolor{lightblue}{cmyk}{0.15,0,0,0}%浅蓝
%--------------------------------------------------------设定标题颜色--------------
\CTEXsetup[format+={\li \color{black}}]{chapter}
\CTEXsetup[format+={\li \color{black}}]{section}
\CTEXsetup[format+={\li \color{black}}]{subsection}

%-----------重定义带编号标题----------------------------
\theoremstyle{plain}
\theoremheaderfont{\hei\rmfamily}
\theorembodyfont{\song\rmfamily}
\theoremseparator{~}%帽号 : 改为 ~ 变空格
\newtheorem{block}{\indent\color{darkblue}}
\newtheorem{definition}{\indent\color{darkblue}定义}[section]
\newtheorem{proposition}{\indent\color{darkblue}命题}[section]
\newtheorem{theorem}{\indent\color{darkblue}定理}[section]
\newtheorem{lemma}{\indent\color{darkblue}引理}[section]
\newtheorem{axiom}{\indent\color{darkblue}公理}[section]
\newtheorem{corollary}{\indent\color{darkblue}推论}[section]
\newtheorem{character}{\indent\color{darkblue}性质}[section]


\newtheorem{exam}{\indent\hei \color{darkblue}例}[chapter]
\newtheorem{exercise}{\indent\hei \color{darkblue}题}
\newtheorem{example}{\hspace{2em}\hei\color{white}Matlab程序}[section]
\newcommand{\program}[1]{\noindent\parbox{14.6cm}{\includegraphics[width=3.1cm]{fig/sjwt.jpg}
		\vspace{-1.12cm}\example\textcolor{darkblue}{\hei#1}}\vspace{0.0cm}}  %实际问题标志



%------------------------------------------重定义不带编号标题----------------------------
\theoremstyle{nonumberplain} \theoremseparator{~}    %帽号 : 改为 ~ 变空格
%\theoremsymbol{$\blacksquare$}                      %在证明结束行尾加上一个黑色方块
\newtheorem{proof}{\indent\hei \textcolor{darkblue}{证明}}
\newtheorem{Solution}{\indent\hei \textcolor{darkblue}{解}}
\newtheorem{REFERENCE}{\indent\hei\sihao \textcolor{darkblue}{参考文献}}
%==========程序框====================================
\newcommand{\programbox}[2]{
	\begin{tcolorbox}[colback=white,colframe=darkblue,title=\example #1]
		#2
\end{tcolorbox}}
%==========程序框结束====================================

%==========命题框====================================
\newcommand{\propositionbox}[1]{\vspace{0.3cm}
	\begin{colorboxed}[boxcolor=darkblue,bgcolor=white]
		\vspace{-0.3cm}\proposition #1
\end{colorboxed}}
%==========命题框结束====================================

%==========引理框====================================
\newcommand{\lemmabox}[1]{\vspace{0.3cm}
	\begin{colorboxed}[boxcolor=darkblue,bgcolor=white]
		\vspace{-0.3cm}\lemma #1
\end{colorboxed}}
%==========引理框结束====================================

%==========定理框====================================
\newcommand{\theorebox}[1]{\vspace{0.3cm}
	\begin{colorboxed}[boxcolor=darkblue,bgcolor=lightblue]
		\vspace{-0.3cm}\theorem #1
\end{colorboxed}}
%==========定理框结束====================================

%==========推论框====================================
\newcommand{\corolbox}[1]{\vspace{0.3cm}
	\begin{colorboxed}[boxcolor=darkblue,bgcolor=white]
		\vspace{-0.3cm}\corollary #1
\end{colorboxed}}
%==========推论框结束====================================

%==========性质框====================================
\newcommand{\charactbox}[1]{\vspace{0.3cm}
	\begin{colorboxed}[boxcolor=darkblue,bgcolor=white]
		\vspace{-0.3cm}\character #1
\end{colorboxed}}
%==========性质框结束====================================

%==========定义框====================================
\newcommand{\definibox}[2]{\vspace{0.3cm}
	\begin{colorboxed}[boxcolor=darkblue,fgcolor=white,bgcolor=darkblue]
		\vspace{-0.4cm}
		\definition\hei\color{white}#1
		\vspace{0.04cm}
	\end{colorboxed}
	\vspace{-0.7cm} \setlength{\fboxsep}{4pt}
	\begin{colorboxed}[boxcolor=darkblue,fgcolor=black,bgcolor=lightblue]
		#2
	\end{colorboxed}\color{black}}
%==========定义框结束====================================

%%-----------------------------------------------------------定义、定理环境-------------------------
%\newcounter{myDefinition}[chapter]\def\themyDefinition{\thechapter.\arabic{myDefinition}}
%\newcounter{myTheorem}[chapter]\def\themyTheorem{\thechapter.\arabic{myTheorem}}
%\newcounter{myCorollary}[chapter]\def\themyCorollary{\thechapter.\arabic{myCorollary}}
%
%\tcbmaketheorem{defi}{定义}{fonttitle=\bfseries\upshape, fontupper=\slshape, arc=0mm, colback=lightblue,colframe=darkblue}{myDefinition}{Definition}
%\tcbmaketheorem{theo}{定理}{fonttitle=\bfseries\upshape, fontupper=\slshape, arc=0mm, colback=lightblue,colframe=darkblue}{myTheorem}{Theorem}
%\tcbmaketheorem{coro}{推论}{fonttitle=\bfseries\upshape, fontupper=\slshape, arc=0mm, colback=lightblue,colframe=darkblue}{myCorollary}{Corollary}
%%------------------------------------------------------------------------------
%\newtheorem{proof}{\indent\hei \textcolor{darkblue}{证明}}
%\newtheorem{Solution}{\indent\hei \textcolor{darkblue}{解}}
%------------------------------------------------定义页眉下单隔线----------------
\newcommand{\makeheadrule}{\makebox[0pt][l]{\color{darkblue}\rule[.7\baselineskip]{\headwidth}{0.3pt}}\vskip-.8\baselineskip}
%-----------------------------------------------定义页眉下双隔线----------------
\makeatletter
\renewcommand{\headrule}{{\if@fancyplain\let\headrulewidth\plainheadrulewidth\fi\makeheadrule}}
\pagestyle{fancy}
\renewcommand{\chaptermark}[1]{\markboth{第\chaptername 章\quad #1}{}}    %去掉章标题中的数字
\renewcommand{\sectionmark}[1]{\markright{\thesection\quad #1}{}}    %去掉节标题中的点
\fancyhf{} %清空页眉
\fancyhead[RO]{\kai{\footnotesize.~\color{darkblue}\thepage~.}}         % 奇数页码显示左边
\fancyhead[LE]{\kai{\footnotesize.~\color{darkblue}\thepage~.}}         % 偶数页码显示右边
\fancyhead[CO]{\song\footnotesize\color{darkblue}\rightmark} % 奇数页码中间显示节标题
\fancyhead[CE]{\song\footnotesize\color{darkblue}\leftmark}  % 偶数页码中间显示章标题
%---------------------------------------------------------------------------------------------------------------------


