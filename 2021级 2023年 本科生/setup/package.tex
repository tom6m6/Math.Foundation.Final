%-------------------------------------宏包引用---------------------------------------------------
\usepackage[paperwidth=185mm,paperheight=230mm,textheight=190mm,textwidth=145mm,left=20mm,right=20mm, top=25mm, bottom=15mm]{geometry}            %定义版面
%--------------------------------------------------------------------------------------
\usepackage{fontspec}
\usepackage{xunicode}
\usepackage{xltxtra}
%--------------------------------------------------------------------------------------
\usepackage[listings,theorems]{tcolorbox}
\usepackage{fancybox}                 % 边框,有阴影,fancybox提供了五种式样\fbox,\shadowbox,\doublebox,\ovalbox,\Ovalbox。
\usepackage{colortbl}                   % 单元格加背景
\usepackage{fancyhdr}                 % 页眉和页脚的相关定义
\usepackage[CJKbookmarks, colorlinks, bookmarksnumbered=true,pdfstartview=FitH,linkcolor=black]{hyperref}   % 书签功能,选项去掉链接红色方框
\usepackage{mathtools}
\usepackage{amssymb}
\usepackage{bm}	% 处理数学公式中的黑斜体的宏包

\usepackage{subfigure}

\usepackage{algorithm}
\usepackage{algorithmic}

\usepackage[amsmath,thmmarks]{ntheorem}         % 定理类环境宏包,其中 amsmath 选项用来兼容 AMS LaTeX 的宏包thmmarks 选项,可以自动恰当地放置定理类环境的结束标记;它还能像图形目录那样生成定理类环境目录

\usepackage{fancyref}                                % 支持引用的宏包