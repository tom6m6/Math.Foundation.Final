\documentclass[12pt,a4paper,openany,twoside]{ctexbook}
%-------------------------------------宏包引用---------------------------------------------------
\usepackage[paperwidth=185mm,paperheight=230mm,textheight=190mm,textwidth=145mm,left=20mm,right=20mm, top=25mm, bottom=15mm]{geometry}            %定义版面
%--------------------------------------------------------------------------------------
\usepackage{fontspec}
\usepackage{xunicode}
\usepackage{xltxtra}
%--------------------------------------------------------------------------------------
\usepackage[listings,theorems]{tcolorbox}
\usepackage{fancybox}                 % 边框,有阴影,fancybox提供了五种式样\fbox,\shadowbox,\doublebox,\ovalbox,\Ovalbox。
\usepackage{colortbl}                   % 单元格加背景
\usepackage{fancyhdr}                 % 页眉和页脚的相关定义
\usepackage[CJKbookmarks, colorlinks, bookmarksnumbered=true,pdfstartview=FitH,linkcolor=black]{hyperref}   % 书签功能,选项去掉链接红色方框
\usepackage{mathtools}
\usepackage{amssymb}
\usepackage{bm}	% 处理数学公式中的黑斜体的宏包

\usepackage{subfigure}

\usepackage{algorithm}
\usepackage{algorithmic}

\usepackage[amsmath,thmmarks]{ntheorem}         % 定理类环境宏包,其中 amsmath 选项用来兼容 AMS LaTeX 的宏包thmmarks 选项,可以自动恰当地放置定理类环境的结束标记;它还能像图形目录那样生成定理类环境目录

\usepackage{fancyref}                                % 支持引用的宏包   %引用宏包所在位置
%-----------------------------------------------------主文档 格式定义---------------------------------
\addtolength{\headsep}{-0.1cm}        %页眉位置
%\addtolength{\footskip}{0.4cm}       %页脚位置
%-----------------------------------------------------设定字体等------------------------------
\setmainfont{Times New Roman}    % 缺省字体
%\setCJKfamilyfont{song}{SimSun}
%\setCJKfamilyfont{hei}{SimHei}
%\setCJKfamilyfont{kai}{KaiTi}
%\setCJKfamilyfont{fs}{FangSong}
%\setCJKfamilyfont{li}{LiSu}
%\setCJKfamilyfont{you}{YouYuan}
%\setCJKfamilyfont{yahei}{Microsoft YaHei}
%\setCJKfamilyfont{xingkai}{STXingkai}
%\setCJKfamilyfont{xinwei}{STXinwei}
%\setCJKfamilyfont{fzyao}{FZYaoTi}
%\setCJKfamilyfont{fzshu}{FZShuTi}
%-------------------------------------------------------------------
\newCJKfontfamily\song{SimSun}
\newCJKfontfamily\hei{SimHei}
\newCJKfontfamily\kai{KaiTi}
\newCJKfontfamily\fs{FangSong}
\newCJKfontfamily\li{LiSu}
\newCJKfontfamily\you{YouYuan}
%\newCJKfontfamily\yahei{Microsoft YaHei}
\newCJKfontfamily\xingkai{STXingkai}
\newCJKfontfamily\xinwei{STXinwei}
\newCJKfontfamily\fzyao{FZYaoTi}
\newCJKfontfamily\fzshu{FZShuTi}
%------------------------------------------------------------------------------------------------------
\newcommand{\linghao}{\fontsize{48pt}{60pt}\selectfont}      % 零号, 1.4倍行距
\newcommand{\chuhao}{\fontsize{42pt}{\baselineskip}\selectfont}  % 初号& 42pt
\newcommand{\xiaochuhao}{\fontsize{36pt}{\baselineskip}\selectfont} % 小初号& 36pt
\newcommand{\yihao}{\fontsize{28pt}{38pt}\selectfont}        % 一号, 1.4倍行距
\newcommand{\erhao}{\fontsize{21pt}{28pt}\selectfont}        % 二号, 1.25倍行距
\newcommand{\xiaoer}{\fontsize{18pt}{18pt}\selectfont}       % 小二, 单倍行距
\newcommand{\sanhao}{\fontsize{16pt}{24pt}\selectfont}       % 三号, 1.5倍行距
\newcommand{\xiaosan}{\fontsize{15pt}{22pt}\selectfont}      % 小三, 1.5倍行距
\newcommand{\sihao}{\fontsize{14pt}{21pt}\selectfont}        % 四号, 1.5倍行距
\newcommand{\xiaosihao}{\fontsize{12pt}{18pt}\selectfont}    % 小四, 1.5倍行距
\newcommand{\dawuhao}{\fontsize{11pt}{11pt}\selectfont}      % 大五号, 单倍行距
\newcommand{\wuhao}{\fontsize{10.5pt}{10.5pt}\selectfont}    % 五号, 单倍行距
\newcommand{\xiaowuhao}{\fontsize{10pt}{10pt}\selectfont}    % 小五号, 单倍行距
\newcommand{\liuhao}{\fontsize{7.875pt}{7.875pt}\selectfont} % 六号, 单倍行距
\newcommand{\qihao}{\fontsize{5.25pt}{\baselineskip}\selectfont}
%--------------------------------------------------------------------------------------------------------
%-----------------------------------------------------------定义颜色---------------
\definecolor{blueblack}{cmyk}{0,0,0,0.35}%浅黑
\definecolor{darkblue}{cmyk}{1,0,0,0}%纯蓝
\definecolor{lightblue}{cmyk}{0.15,0,0,0}%浅蓝
%--------------------------------------------------------设定标题颜色--------------
\CTEXsetup[format+={\li \color{black}}]{chapter}
\CTEXsetup[format+={\li \color{black}}]{section}
\CTEXsetup[format+={\li \color{black}}]{subsection}

%-----------重定义带编号标题----------------------------
\theoremstyle{plain}
\theoremheaderfont{\hei\rmfamily}
\theorembodyfont{\song\rmfamily}
\theoremseparator{~}%帽号 : 改为 ~ 变空格
\newtheorem{block}{\indent\color{darkblue}}
\newtheorem{definition}{\indent\color{darkblue}定义}[section]
\newtheorem{proposition}{\indent\color{darkblue}命题}[section]
\newtheorem{theorem}{\indent\color{darkblue}定理}[section]
\newtheorem{lemma}{\indent\color{darkblue}引理}[section]
\newtheorem{axiom}{\indent\color{darkblue}公理}[section]
\newtheorem{corollary}{\indent\color{darkblue}推论}[section]
\newtheorem{character}{\indent\color{darkblue}性质}[section]


\newtheorem{exam}{\indent\hei \color{darkblue}例}[chapter]
\newtheorem{exercise}{\indent\hei \color{darkblue}题}
\newtheorem{example}{\hspace{2em}\hei\color{white}Matlab程序}[section]
\newcommand{\program}[1]{\noindent\parbox{14.6cm}{\includegraphics[width=3.1cm]{fig/sjwt.jpg}
		\vspace{-1.12cm}\example\textcolor{darkblue}{\hei#1}}\vspace{0.0cm}}  %实际问题标志



%------------------------------------------重定义不带编号标题----------------------------
\theoremstyle{nonumberplain} \theoremseparator{~}    %帽号 : 改为 ~ 变空格
%\theoremsymbol{$\blacksquare$}                      %在证明结束行尾加上一个黑色方块
\newtheorem{proof}{\indent\hei \textcolor{darkblue}{证明}}
\newtheorem{Solution}{\indent\hei \textcolor{darkblue}{解}}
\newtheorem{REFERENCE}{\indent\hei\sihao \textcolor{darkblue}{参考文献}}
%==========程序框====================================
\newcommand{\programbox}[2]{
	\begin{tcolorbox}[colback=white,colframe=darkblue,title=\example #1]
		#2
\end{tcolorbox}}
%==========程序框结束====================================

%==========命题框====================================
\newcommand{\propositionbox}[1]{\vspace{0.3cm}
	\begin{colorboxed}[boxcolor=darkblue,bgcolor=white]
		\vspace{-0.3cm}\proposition #1
\end{colorboxed}}
%==========命题框结束====================================

%==========引理框====================================
\newcommand{\lemmabox}[1]{\vspace{0.3cm}
	\begin{colorboxed}[boxcolor=darkblue,bgcolor=white]
		\vspace{-0.3cm}\lemma #1
\end{colorboxed}}
%==========引理框结束====================================

%==========定理框====================================
\newcommand{\theorebox}[1]{\vspace{0.3cm}
	\begin{colorboxed}[boxcolor=darkblue,bgcolor=lightblue]
		\vspace{-0.3cm}\theorem #1
\end{colorboxed}}
%==========定理框结束====================================

%==========推论框====================================
\newcommand{\corolbox}[1]{\vspace{0.3cm}
	\begin{colorboxed}[boxcolor=darkblue,bgcolor=white]
		\vspace{-0.3cm}\corollary #1
\end{colorboxed}}
%==========推论框结束====================================

%==========性质框====================================
\newcommand{\charactbox}[1]{\vspace{0.3cm}
	\begin{colorboxed}[boxcolor=darkblue,bgcolor=white]
		\vspace{-0.3cm}\character #1
\end{colorboxed}}
%==========性质框结束====================================

%==========定义框====================================
\newcommand{\definibox}[2]{\vspace{0.3cm}
	\begin{colorboxed}[boxcolor=darkblue,fgcolor=white,bgcolor=darkblue]
		\vspace{-0.4cm}
		\definition\hei\color{white}#1
		\vspace{0.04cm}
	\end{colorboxed}
	\vspace{-0.7cm} \setlength{\fboxsep}{4pt}
	\begin{colorboxed}[boxcolor=darkblue,fgcolor=black,bgcolor=lightblue]
		#2
	\end{colorboxed}\color{black}}
%==========定义框结束====================================

%%-----------------------------------------------------------定义、定理环境-------------------------
%\newcounter{myDefinition}[chapter]\def\themyDefinition{\thechapter.\arabic{myDefinition}}
%\newcounter{myTheorem}[chapter]\def\themyTheorem{\thechapter.\arabic{myTheorem}}
%\newcounter{myCorollary}[chapter]\def\themyCorollary{\thechapter.\arabic{myCorollary}}
%
%\tcbmaketheorem{defi}{定义}{fonttitle=\bfseries\upshape, fontupper=\slshape, arc=0mm, colback=lightblue,colframe=darkblue}{myDefinition}{Definition}
%\tcbmaketheorem{theo}{定理}{fonttitle=\bfseries\upshape, fontupper=\slshape, arc=0mm, colback=lightblue,colframe=darkblue}{myTheorem}{Theorem}
%\tcbmaketheorem{coro}{推论}{fonttitle=\bfseries\upshape, fontupper=\slshape, arc=0mm, colback=lightblue,colframe=darkblue}{myCorollary}{Corollary}
%%------------------------------------------------------------------------------
%\newtheorem{proof}{\indent\hei \textcolor{darkblue}{证明}}
%\newtheorem{Solution}{\indent\hei \textcolor{darkblue}{解}}
%------------------------------------------------定义页眉下单隔线----------------
\newcommand{\makeheadrule}{\makebox[0pt][l]{\color{darkblue}\rule[.7\baselineskip]{\headwidth}{0.3pt}}\vskip-.8\baselineskip}
%-----------------------------------------------定义页眉下双隔线----------------
\makeatletter
\renewcommand{\headrule}{{\if@fancyplain\let\headrulewidth\plainheadrulewidth\fi\makeheadrule}}
\pagestyle{fancy}
\renewcommand{\chaptermark}[1]{\markboth{第\chaptername 章\quad #1}{}}    %去掉章标题中的数字
\renewcommand{\sectionmark}[1]{\markright{\thesection\quad #1}{}}    %去掉节标题中的点
\fancyhf{} %清空页眉
\fancyhead[RO]{\kai{\footnotesize.~\color{darkblue}\thepage~.}}         % 奇数页码显示左边
\fancyhead[LE]{\kai{\footnotesize.~\color{darkblue}\thepage~.}}         % 偶数页码显示右边
\fancyhead[CO]{\song\footnotesize\color{darkblue}\rightmark} % 奇数页码中间显示节标题
\fancyhead[CE]{\song\footnotesize\color{darkblue}\leftmark}  % 偶数页码中间显示章标题
%---------------------------------------------------------------------------------------------------------------------


    %格式所在位置
% !Mode:: "TeX:UTF-8"
\newcommand{\normmm}[1]{{\left\vert\kern-0.25ex\left\vert\kern-0.25ex\left\vert #1 
		\right\vert\kern-0.25ex\right\vert\kern-0.25ex\right\vert}}

\newcommand{\argmax}{\arg\max}
\newcommand{\argmin}{\arg\min}
\newcommand{\sigmoid}{\text{sigmoid}}
\newcommand{\norm}[1]{\left\lVert#1\right\rVert}
\newcommand{\inprod}[2]{\langle#1,#2\rangle}
\newcommand{\Tr}{\text{Tr}}
\renewcommand{\d}{\text{d}}

\newcommand{\Var}{\text{Var}}
\newcommand{\Cov}{\text{Cov}}
\newcommand{\plim}{\text{plim}}
\newcommand{\Tsp}{\top}

\newcommand{\diag}{\mathop{\rm diag}\nolimits}
\newcommand{\rank}{\mathop{\rm rank}\nolimits}
\newcommand{\Laplace}{\mathop{\rm Laplace}\nolimits}
\newcommand{\Bernoulli}{\mathop{\rm Bernoulli}\nolimits}
\newcommand{\Uniform}{\mathop{\rm Uniform}\nolimits}
\newcommand{\Possion}{\mathop{\rm Possion}\nolimits}
\newcommand{\Ber}{\mathop{\rm Ber}\nolimits}
\newcommand{\Bin}{\mathop{\rm Bin}\nolimits}
\newcommand{\Cat}{\mathop{\rm Cat}\nolimits}
\newcommand{\Dir}{\mathop{\rm Dir}\nolimits}
\newcommand{\Lap}{\mathop{\rm Lap}\nolimits}
\newcommand{\sigm}{\mathop{\rm sigm}\nolimits}

\newcommand{\RSS}{\mathop{\rm RSS}\nolimits}
\newcommand{\se}{\mathop{\rm se}\nolimits}
\newcommand{\logit}{\mathop{\rm logit}\nolimits}
\newcommand{\dom}{\mathop{\bf dom}\nolimits}
\newcommand{\epi}{\mathop{\bf epi}\nolimits}
\newcommand{\hypo}{\mathop{\bf hypo}\nolimits}
\newcommand{\prob}{\mathop{\bf prob}\nolimits}
\newcommand{\relint}{\mathop{\bf relint}\nolimits}
\newcommand{\aff}{\mathop{\bf aff}\nolimits}
\newcommand{\spn}{\mathop{\bf span}\nolimits}
% Scala 标量 暂不使用
\newcommand{\Sa}{\mathit{a}}
\newcommand{\Sb}{\mathit{b}}
\newcommand{\Sc}{\mathit{c}}
\newcommand{\Sd}{\mathit{d}}
\newcommand{\Se}{\mathit{e}}
\newcommand{\Sf}{\mathit{f}}
\newcommand{\Sg}{\mathit{g}}
\newcommand{\Sh}{\mathit{h}}
\newcommand{\Si}{\mathit{i}}
\newcommand{\Sj}{\mathit{j}}
\newcommand{\Sk}{\mathit{k}}
\newcommand{\Sl}{\mathit{l}}
\newcommand{\Sm}{\mathit{m}}
\newcommand{\Sn}{\mathit{n}}
\newcommand{\So}{\mathit{o}}
\newcommand{\Sp}{\mathit{p}}
\newcommand{\Sq}{\mathit{q}}
\newcommand{\Sr}{\mathit{r}}
\newcommand{\Ss}{\mathit{s}}
\newcommand{\St}{\mathit{t}}
\newcommand{\Su}{\mathit{u}}
\newcommand{\Sv}{\mathit{v}}
\newcommand{\Sw}{\mathit{w}}
\newcommand{\Sx}{\mathit{x}}
\newcommand{\Sy}{\mathit{y}}
\newcommand{\Sz}{\mathit{z}}

\newcommand{\SA}{\mathit{A}}
\newcommand{\SB}{\mathit{B}}
\newcommand{\SC}{\mathit{C}}
\newcommand{\SD}{\mathit{D}}
\newcommand{\SE}{\mathit{E}}
\newcommand{\SF}{\mathit{F}}
\newcommand{\SG}{\mathit{G}}
\newcommand{\SH}{\mathit{H}}
\newcommand{\SJ}{\mathit{J}}
\newcommand{\SK}{\mathit{K}}
\newcommand{\SI}{\mathit{L}}
\newcommand{\SM}{\mathit{M}}
\newcommand{\SN}{\mathit{N}}
\newcommand{\SO}{\mathit{O}}
\newcommand{\SP}{\mathit{P}}
\newcommand{\SQ}{\mathit{Q}}
\newcommand{\SR}{\mathit{R}}
\newcommand{\ST}{\mathit{T}}
\newcommand{\SU}{\mathit{U}}
\newcommand{\SV}{\mathit{V}}
\newcommand{\SW}{\mathit{W}}
\newcommand{\SX}{\mathit{X}}
\newcommand{\SY}{\mathit{Y}}
\newcommand{\SZ}{\mathit{Z}}



% Vector 向量 \VI 全1向量 \VO 零向量
\newcommand{\Va}{\boldsymbol{\mathit{a}}}
\newcommand{\Vb}{\boldsymbol{\mathit{b}}}
\newcommand{\Vc}{\boldsymbol{\mathit{c}}}
\newcommand{\Vd}{\boldsymbol{\mathit{d}}}
\newcommand{\Ve}{\boldsymbol{\mathit{e}}}
\newcommand{\Vf}{\boldsymbol{\mathit{f}}}
\newcommand{\Vg}{\boldsymbol{\mathit{g}}}
\newcommand{\Vh}{\boldsymbol{\mathit{h}}}
\newcommand{\Vi}{\boldsymbol{\mathit{i}}}
\newcommand{\Vj}{\boldsymbol{\mathit{j}}}
\newcommand{\Vk}{\boldsymbol{\mathit{k}}}
\newcommand{\Vl}{\boldsymbol{\mathit{l}}}
\newcommand{\Vm}{\boldsymbol{\mathit{m}}}
\newcommand{\Vn}{\boldsymbol{\mathit{n}}}
\newcommand{\Vo}{\boldsymbol{\mathit{o}}}
\newcommand{\Vp}{\boldsymbol{\mathit{p}}}
\newcommand{\Vq}{\boldsymbol{\mathit{q}}}
\newcommand{\Vr}{\boldsymbol{\mathit{r}}}
\newcommand{\Vs}{\boldsymbol{\mathit{s}}}
\newcommand{\Vt}{\boldsymbol{\mathit{t}}}
\newcommand{\Vu}{\boldsymbol{\mathit{u}}}
\newcommand{\Vv}{\boldsymbol{\mathit{v}}}
\newcommand{\Vw}{\boldsymbol{\mathit{w}}}
\newcommand{\Vx}{\boldsymbol{\mathit{x}}}
\newcommand{\Vy}{\boldsymbol{\mathit{y}}}
\newcommand{\Vz}{\boldsymbol{\mathit{z}}}

%零向量,由于命令中不能出现数字,用字母“O”代替数字“0”
\newcommand{\VO}{\boldsymbol{0}}
\newcommand{\VI}{\boldsymbol{1}}

% Matrix 矩阵
\newcommand{\MA}{\boldsymbol{\mathit{A}}}
\newcommand{\MB}{\boldsymbol{\mathit{B}}}
\newcommand{\MC}{\boldsymbol{\mathit{C}}}
\newcommand{\MD}{\boldsymbol{\mathit{D}}}
\newcommand{\ME}{\boldsymbol{\mathit{E}}}
\newcommand{\MF}{\boldsymbol{\mathit{F}}}
\newcommand{\MG}{\boldsymbol{\mathit{G}}}
\newcommand{\MH}{\boldsymbol{\mathit{H}}}
\newcommand{\MI}{\boldsymbol{\mathit{I}}}
\newcommand{\MJ}{\boldsymbol{\mathit{J}}}
\newcommand{\MK}{\boldsymbol{\mathit{K}}}
\newcommand{\ML}{\boldsymbol{\mathit{L}}}
\newcommand{\MM}{\boldsymbol{\mathit{M}}}
\newcommand{\MN}{\boldsymbol{\mathit{N}}}
\newcommand{\MO}{\boldsymbol{\mathit{O}}}
\newcommand{\MP}{\boldsymbol{\mathit{P}}}
\newcommand{\MQ}{\boldsymbol{\mathit{Q}}}
\newcommand{\MR}{\boldsymbol{\mathit{R}}}
\newcommand{\MS}{\boldsymbol{\mathit{S}}}
\newcommand{\MT}{\boldsymbol{\mathit{T}}}
\newcommand{\MU}{\boldsymbol{\mathit{U}}}
\newcommand{\MV}{\boldsymbol{\mathit{V}}}
\newcommand{\MW}{\boldsymbol{\mathit{W}}}
\newcommand{\MX}{\boldsymbol{\mathit{X}}}
\newcommand{\MY}{\boldsymbol{\mathit{Y}}}
\newcommand{\MZ}{\boldsymbol{\mathit{Z}}}


%Tensor 张量
\newcommand{\TSA}{\textsf{\textbf{A}}}
\newcommand{\TSB}{\textsf{\textbf{B}}}
\newcommand{\TSC}{\textsf{\textbf{C}}}
\newcommand{\TSD}{\textsf{\textbf{D}}}
\newcommand{\TSE}{\textsf{\textbf{E}}}
\newcommand{\TSF}{\textsf{\textbf{F}}}
\newcommand{\TSG}{\textsf{\textbf{G}}}
\newcommand{\TSH}{\textsf{\textbf{H}}}
\newcommand{\TSI}{\textsf{\textbf{I}}}
\newcommand{\TSJ}{\textsf{\textbf{J}}}
\newcommand{\TSK}{\textsf{\textbf{K}}}
\newcommand{\TSL}{\textsf{\textbf{L}}}
\newcommand{\TSM}{\textsf{\textbf{M}}}
\newcommand{\TSN}{\textsf{\textbf{N}}}
\newcommand{\TSO}{\textsf{\textbf{O}}}
\newcommand{\TSP}{\textsf{\textbf{P}}}
\newcommand{\TSQ}{\textsf{\textbf{Q}}}
\newcommand{\TSR}{\textsf{\textbf{R}}}
\newcommand{\TSS}{\textsf{\textbf{S}}}
\newcommand{\TST}{\textsf{\textbf{T}}}
\newcommand{\TSU}{\textsf{\textbf{U}}}
\newcommand{\TSV}{\textsf{\textbf{V}}}
\newcommand{\TSW}{\textsf{\textbf{W}}}
\newcommand{\TSX}{\textsf{\textbf{X}}}
\newcommand{\TSY}{\textsf{\textbf{Y}}}
\newcommand{\TSZ}{\textsf{\textbf{Z}}}

% Tensor Element
\newcommand{\TEA}{\textit{\textsf{A}}}
\newcommand{\TEB}{\textit{\textsf{B}}}
\newcommand{\TEC}{\textit{\textsf{C}}}
\newcommand{\TED}{\textit{\textsf{D}}}
\newcommand{\TEE}{\textit{\textsf{E}}}
\newcommand{\TEF}{\textit{\textsf{F}}}
\newcommand{\TEG}{\textit{\textsf{G}}}
\newcommand{\TEH}{\textit{\textsf{H}}}
\newcommand{\TEI}{\textit{\textsf{I}}}
\newcommand{\TEJ}{\textit{\textsf{J}}}
\newcommand{\TEK}{\textit{\textsf{K}}}
\newcommand{\TEL}{\textit{\textsf{L}}}
\newcommand{\TEM}{\textit{\textsf{M}}}
\newcommand{\TEN}{\textit{\textsf{N}}}
\newcommand{\TEO}{\textit{\textsf{O}}}
\newcommand{\TEP}{\textit{\textsf{P}}}
\newcommand{\TEQ}{\textit{\textsf{Q}}}
\newcommand{\TER}{\textit{\textsf{R}}}
\newcommand{\TES}{\textit{\textsf{S}}}
\newcommand{\TET}{\textit{\textsf{T}}}
\newcommand{\TEU}{\textit{\textsf{U}}}
\newcommand{\TEV}{\textit{\textsf{V}}}
\newcommand{\TEW}{\textit{\textsf{W}}}
\newcommand{\TEX}{\textit{\textsf{X}}}
\newcommand{\TEY}{\textit{\textsf{Y}}}
\newcommand{\TEZ}{\textit{\textsf{Z}}}

% Random Scala
\newcommand{\RSa}{\mathrm{a}}
\newcommand{\RSb}{\mathrm{b}}
\newcommand{\RSc}{\mathrm{c}}
\newcommand{\RSd}{\mathrm{d}}
\newcommand{\RSe}{\mathrm{e}}
\newcommand{\RSf}{\mathrm{f}}
\newcommand{\RSg}{\mathrm{g}}
\newcommand{\RSh}{\mathrm{h}}
\newcommand{\RSi}{\mathrm{i}}
\newcommand{\RSj}{\mathrm{j}}
\newcommand{\RSk}{\mathrm{k}}
\newcommand{\RSl}{\mathrm{l}}
\newcommand{\RSm}{\mathrm{m}}
\newcommand{\RSn}{\mathrm{n}}
\newcommand{\RSo}{\mathrm{o}}
\newcommand{\RSp}{\mathrm{p}}
\newcommand{\RSq}{\mathrm{q}}
\newcommand{\RSr}{\mathrm{r}}
\newcommand{\RSs}{\mathrm{s}}
\newcommand{\RSt}{\mathrm{t}}
\newcommand{\RSu}{\mathrm{u}}
\newcommand{\RSv}{\mathrm{v}}
\newcommand{\RSw}{\mathrm{w}}
\newcommand{\RSx}{\mathrm{x}}
\newcommand{\RSy}{\mathrm{y}}
\newcommand{\RSz}{\mathrm{z}}


% Random Vector
\newcommand{\RVa}{\mathbf{a}}
\newcommand{\RVb}{\mathbf{b}}
\newcommand{\RVc}{\mathbf{c}}
\newcommand{\RVd}{\mathbf{d}}
\newcommand{\RVe}{\mathbf{e}}
\newcommand{\RVf}{\mathbf{f}}
\newcommand{\RVg}{\mathbf{g}}
\newcommand{\RVh}{\mathbf{h}}
\newcommand{\RVi}{\mathbf{i}}
\newcommand{\RVj}{\mathbf{j}}
\newcommand{\RVk}{\mathbf{k}}
\newcommand{\RVl}{\mathbf{l}}
\newcommand{\RVm}{\mathbf{m}}
\newcommand{\RVn}{\mathbf{n}}
\newcommand{\RVo}{\mathbf{o}}
\newcommand{\RVp}{\mathbf{p}}
\newcommand{\RVq}{\mathbf{q}}
\newcommand{\RVr}{\mathbf{r}}
\newcommand{\RVs}{\mathbf{s}}
\newcommand{\RVt}{\mathbf{t}}
\newcommand{\RVu}{\mathbf{u}}
\newcommand{\RVv}{\mathbf{v}}
\newcommand{\RVw}{\mathbf{w}}
\newcommand{\RVx}{\mathbf{x}}
\newcommand{\RVy}{\mathbf{y}}
\newcommand{\RVz}{\mathbf{z}}

% Random Matrix
\newcommand{\RMX}{\boldsymbol{\mathrm{X}}}
\newcommand{\RMA}{\boldsymbol{\mathrm{A}}}

\newcommand{\Valpha}{\boldsymbol{\alpha}}
\newcommand{\Vbeta}{\boldsymbol{\beta}}
\newcommand{\Vtheta}{\boldsymbol{\theta}}
\newcommand{\Vlambda}{\boldsymbol{\lambda}}
\newcommand{\Veta}{\boldsymbol{\eta}}
\newcommand{\Vepsilon}{\boldsymbol{\varepsilon}}
\newcommand{\Vmu}{\boldsymbol{\mu}}
\newcommand{\VPhi}{\boldsymbol{\Phi}}
\newcommand{\Vsigma}{\boldsymbol{\sigma}}

\newcommand{\Vrho}{\boldsymbol{\rho}}
\newcommand{\Vgamma}{\boldsymbol{\gamma}}
\newcommand{\Vomega}{\boldsymbol{\omega}}
\newcommand{\Vpsi}{\boldsymbol{\psi}}
\newcommand{\Vzeta}{\boldsymbol{\zeta}}
\newcommand{\Vxi}{\boldsymbol{\xi}}
\newcommand{\Vone}{\boldsymbol{1}}
\newcommand{\Vvarphi}{\boldsymbol{\varphi}}
\newcommand{\MLambda}{\boldsymbol{\Lambda}}
\newcommand{\MSigma}{\boldsymbol{\Sigma}}

\newcommand{\MPi}{\boldsymbol{\mathit{\Pi}}}

%线性变换 书法字体 \CalA
\newcommand{\CalA}{\mathcal{A}}
\newcommand{\CalB}{\mathcal{B}}
\newcommand{\CalC}{\mathcal{C}}
\newcommand{\CalD}{\mathcal{D}}
\newcommand{\CalE}{\mathcal{E}}
\newcommand{\CalF}{\mathcal{F}}
\newcommand{\CalG}{\mathcal{G}}
\newcommand{\CalH}{\mathcal{H}}
\newcommand{\CalI}{\mathcal{I}}
\newcommand{\CalJ}{\mathcal{J}}
\newcommand{\CalK}{\mathcal{K}}
\newcommand{\CalL}{\mathcal{L}}
\newcommand{\CalM}{\mathcal{M}}
\newcommand{\CalN}{\mathcal{N}}
\newcommand{\CalO}{\mathcal{O}}
\newcommand{\CalP}{\mathcal{P}}
\newcommand{\CalQ}{\mathcal{Q}}
\newcommand{\CalR}{\mathcal{R}}
\newcommand{\CalS}{\mathcal{S}}
\newcommand{\CalT}{\mathcal{T}}
\newcommand{\CalU}{\mathcal{U}}
\newcommand{\CalV}{\mathcal{V}}
\newcommand{\CalW}{\mathcal{W}}
\newcommand{\CalX}{\mathcal{X}}
\newcommand{\CalY}{\mathcal{Y}}
\newcommand{\CalZ}{\mathcal{Z}}


% Set
\newcommand{\SetA}{\mathbb{A}}
\newcommand{\SetB}{\mathbb{B}}
\newcommand{\SetC}{\mathbb{C}}
\newcommand{\SetD}{\mathbb{D}}
\newcommand{\SetE}{\mathbb{E}}
\newcommand{\SetF}{\mathbb{F}}
\newcommand{\SetG}{\mathbb{G}}
\newcommand{\SetH}{\mathbb{H}}
\newcommand{\SetI}{\mathbb{I}}
\newcommand{\SetJ}{\mathbb{J}}
\newcommand{\SetK}{\mathbb{K}}
\newcommand{\SetL}{\mathbb{L}}
\newcommand{\SetM}{\mathbb{M}}
\newcommand{\SetN}{\mathbb{N}}
\newcommand{\SetO}{\mathbb{O}}
\newcommand{\SetP}{\mathbb{P}}
\newcommand{\SetQ}{\mathbb{Q}}
\newcommand{\SetR}{\mathbb{R}}
\newcommand{\SetS}{\mathbb{S}}
\newcommand{\SetT}{\mathbb{T}}
\newcommand{\SetU}{\mathbb{U}}
\newcommand{\SetV}{\mathbb{V}}
\newcommand{\SetW}{\mathbb{W}}
\newcommand{\SetX}{\mathbb{X}}
\newcommand{\SetY}{\mathbb{Y}}
\newcommand{\SetZ}{\mathbb{Z}}


%Number Set
\newcommand{\NN}{\mathbb{N}}
\newcommand{\NZ}{\mathbb{Z}}
\newcommand{\NQ}{\mathbb{Q}}
\newcommand{\NR}{\mathbb{R}}
\newcommand{\NC}{\mathbb{C}}
\newcommand{\NK}{\mathbb{K}}

%转置
\renewcommand{\T}{^{\mathrm{T}}}


\newcommand{\Col}[1]{\text{Col}\left(#1\right)}
\newcommand{\Row}[1]{\text{Row}\left(#1\right)}
\newcommand{\Range}[1]{\text{Range}\left(#1\right)}
\newcommand{\Null}[1]{\text{Null}\left(#1\right)}
\newcommand{\nullity}[1]{\text{nullity}\left(#1\right)}
\newcommand{\Ker}[1]{\text{Ker}\left(#1\right)}
\let\olddefinition\definition
\let\oldproposition\proposition
\let\oldtheorem\theorem
\let\oldlemma\lemma
\let\oldaxiom\axiom
\let\oldcorollary\corollary
\let\oldcharacter\character
\let\oldexercise\exercise
\let\oldexam\exam
\let\oldexample\example

\let\oldequation\equation

%\renewcommand{\definition}{\olddefinition\label{dy:\thedefinition}}
%\renewcommand{\proposition}{\oldproposition\label{mt:\theproposition}}
%\renewcommand{\theorem}{\oldtheorem\label{dl:\thetheorem}}
%\renewcommand{\lemma}{\oldlemma\label{yl:\thelemma}}
%\renewcommand{\axiom}{\oldaxiom\label{gl:\theaxiom}}
%\renewcommand{\corollary}{\oldcorollary\label{tl:\thecorollary}}
%\renewcommand{\character}{\oldcharacter\label{xz:\thecharacter}}
%\renewcommand{\exercise}{\oldexercise\label{xt:\theexercise}}
%\renewcommand{\exam}{\oldexam\label{li:\theexam}}
%\renewcommand{\example}{\oldexample\label{cx:\theexample}}

%\renewcommand{\equation}{\oldequation\label{gs:\theequation}}
%\newcommand{\gslbl}[3]{\label{gs:#1:#2:#3}}
%\newcommand{\gsref}[3]{式\ref{gs:#1:#2:#3}}
%\newcommand{\gslast}{式\ref{gs:\theequation}}

%从此处开始看
%标签
\newcommand{\dylbl}[1]{\label{dy:#1}}
\newcommand{\mtlbl}[1]{\label{mt:#1}}
\newcommand{\dllbl}[1]{\label{dl:#1}}
\newcommand{\yllbl}[1]{\label{yl:#1}}
\newcommand{\gllbl}[1]{\label{gl:#1}}
\newcommand{\tllbl}[1]{\label{tl:#1}}
\newcommand{\xzlbl}[1]{\label{xz:#1}}
\newcommand{\xtlbl}[1]{\label{xt:#1}}
\newcommand{\lilbl}[1]{\label{li:#1}}
\newcommand{\cxlbl}[1]{\label{cx:#1}}

%引用
\newcommand{\dyref}[1]{定义\ref{dy:#1}}
\newcommand{\mtref}[1]{命题\ref{mt:#1}}
\newcommand{\dlref}[1]{定理\ref{dl:#1}}
\newcommand{\ylref}[1]{引理\ref{yl:#1}}
\newcommand{\glref}[1]{公理\ref{gl:#1}}
\newcommand{\tlref}[1]{推论\ref{tl:#1}}
\newcommand{\xzref}[1]{性质\ref{xz:#1}}
\newcommand{\xtref}[1]{习题\ref{xt:#1}}
\newcommand{\liref}[1]{例 \ref{li:#1}}
\newcommand{\cxref}[1]{程序\ref{cx:#1}}

%引用最后一个,只能在本节内部,可以不打标签
\newcommand{\dylast}[1][0]{\addtocounter{definition}{-#1}定义\ref{dy:\thedefinition}%
	\addtocounter{definition}{#1}}
\newcommand{\mtlast}[1][0]{\addtocounter{proposition}{-#1}命题\ref{mt:\theproposition}%
	\addtocounter{proposition}{#1}}
\newcommand{\dllast}[1][0]{\addtocounter{theorem}{-#1}定理\ref{dl:\thetheorem}%
	\addtocounter{theorem}{#1}}
\newcommand{\yllast}[1][0]{\addtocounter{lemma}{-#1}引理\ref{yl:\thelemma}%
	\addtocounter{lemma}{#1}}
\newcommand{\gllast}[1][0]{\addtocounter{axiom}{-#1}公理\ref{gl:\theaxiom}%
	\addtocounter{axiom}{#1}}
\newcommand{\tllast}[1][0]{\addtocounter{corollary}{-#1}推论\ref{tl:\thecorollary}%
	\addtocounter{corollary}{#1}}
\newcommand{\xzlast}[1][0]{\addtocounter{character}{-#1}性质\ref{xz:\thecharacter}%
	\addtocounter{character}{#1}}
\newcommand{\xtlast}[1][0]{\addtocounter{exercise}{-#1}习题\ref{xt:\theexercise}%
	\addtocounter{exercise}{#1}}
\newcommand{\lilast}[1][0]{\addtocounter{exam}{-#1}例  \ref{li:\theexam}%
	\addtocounter{exam}{#1}}
\newcommand{\cxlast}[1][0]{\addtocounter{example}{-#1}程序\ref{cx:\theexample}%
	\addtocounter{example}{#1}}

%引用最后一个公式,只能在本章内部,没有超链接,可以不打标签

\newcommand{\gslast}[1][0]{\addtocounter{equation}{-#1}%
	(\theequation)\addtocounter{equation}{#1}}

\newcommand{\underbox}[2]{\kern0pt\underline{\makebox[#1]{#2}}\kern0pt\relax}
\newcommand{\purespace}[1]{\kern0pt {\hspace{#1}}\kern0pt\relax}

\begin{document}
	
	\setcounter{chapter}{1}
	\chapter*{华东师范大学期末试卷 (A 卷) 答案}
	\vspace{-1cm}
	\section*{2023—2024学年第一学期}
	%\begin{center}
	%	学院:数据科学与工程学院 \qquad 考试形式:闭卷 \qquad 所需时间:120分钟
	%\end{center}
	\begin{center}
		考试科目:\underbox{11\ccwd}{数据科学与工程数学基础} \quad 任课教师:\underbox{6\ccwd}{黄定江} \\
		姓\purespace{2\ccwd}名:\underbox{11\ccwd}{} \quad 学\purespace{2\ccwd}号:\underbox{6\ccwd}{} \\
		专\purespace{2\ccwd}业:\underbox{11\ccwd}{} \quad 班\purespace{2\ccwd}级:\underbox{6\ccwd}{}\\
	\end{center}
	%\begin{center}
	%	考生姓名:\underline{~~~~~~}\underline{~~~~~~}\qquad 学号:\underline{~~~~~~}\underline{~~~~~~}\qquad 专业:\underline{~~~~~~}\underline{~~~~~~}\qquad 任课教师:黄定江
	%\end{center}
	{ \begin{center}
			\begin{tabular}{|c|c|c|c|c|c|c|c|c|c|c|}
				\hline 
				题号 & 一 & 二 & 三 & 四 & 五 & 六 & 七 & 八&总分 &阅卷人签名 \\ 
				\hline 
				%			满分 & 13 & 10 & 13 & 13 & 12 & 12 & 13 & 14 & 100  \\ 
				%			\hline 
				得分 &  &  &  &  &  &  &  & & & \\ 
				\hline 
				%			阅卷人 &  &  &  &  &  &  &  & &  \\ 
				%			\hline 
			\end{tabular} 
			
	\end{center}}


\begin{exercise}(13')
	完成以下问题:
	
	\begin{itemize}
		\item [(1)]  对矩阵$\MM = \begin{pmatrix}
			5& 3& -1&  3\\
			0& 1&  1& -2\\
			-5& -3&  4& -4\\
			0&  1&  1&  0\\
		\end{pmatrix}$进行LU分解;
		\item [(2)] 设 $A$ 对称且 $a_{11} \neq 0$, 并假 经过一步 Gauss 消去之后, $A$ 具有如下形式
		$$
		\left[\begin{array}{ll}
			a_{11} & a_1^{\mathrm{T}} \\
			\VO & A_2
		\end{array}\right]
		$$
		证明:$A_2$仍是对称矩阵。
	\end{itemize}
	
\end{exercise}
\begin{Solution}
	(1)\begin{center}
		$\MA=\begin{pmatrix}
			5& 3& -1&  3\\
			0& 1&  1& -2\\
			-5& -3&  4& -4\\
			0&  1&  1&  0\\
		\end{pmatrix}
		=
		\begin{pmatrix}
			1& 0& 0&  0\\
			0& 1&  0& 0\\
			-1& 0&  1& 0\\
			0&  1&  0&  1\\
		\end{pmatrix}
		\begin{pmatrix}
			5& 3& -1&  3\\
			0& 1&  1& -2\\
			0& 0&  3& -1\\
			0&  0&  0&  2\\
		\end{pmatrix}
		=\ML\MU$
	\end{center}
	\hfill \textcolor{red}{\textbf{(7分)}}
	
	(2)记矩阵$A$和高斯变换矩阵分别为:
	$$
	A = \left[\begin{array}{ll}
		a_{11} & a_1^{\mathrm{T}} \\
		a_1 & A_1
	\end{array}\right],
	G = \left[\begin{array}{ll}
		1 & \VO^{\mathrm{T}} \\
		-l_1 & I
	\end{array}\right]
	$$
	则
	$$
	GA=\left[\begin{array}{ll}
		a_{11} & a_1^{\mathrm{T}} \\
		-a_{11} l_1+a_1 & -l_1 a_1^{\mathrm{T}} + A_1
	\end{array}\right]
	$$
	易知$-a_{11} l_1+a_1=\VO$以及$A_2=-l_1 a_1^{\mathrm{T}} + A_1$。由第一个等式可得$l_1=1/a_{11} a_1$,代入第二个等式知
	$$A_2=-\frac{1}{a_{11}} a_1 a_1^{\mathrm{T}} + A_1.$$
	故可知$A_2$仍是对称矩阵。 \hfill \textcolor{red}{\textbf{(6分)}}
\end{Solution}


\begin{exercise}(10')
	$ \NR^5 $的欧氏空间。子空间$U \subseteq \mathbb{R}^{5}$ 和 $\Vx  \in \mathbb{R}^{5}$如下:
	\[
	U=\spn \left\{ \left[\begin{array}{c}{0} \\ {-1} \\ {2} \\ {0} \\ {1}\end{array}\right],\left[\begin{array}{c}{1} \\ {-1} \\ {1} \\ {-1} \\ {1}\end{array}\right]
%	,\left[\begin{array}{c}{-3} \\ {4} \\ {1} \\ {2} \\ {1}\end{array}\right],\left[\begin{array}{c}{-1} \\ {-3} \\ {5} \\ {0} \\ {7}\end{array}\right]
	\right\}, \quad \boldsymbol{x}=\left[\begin{array}{c}{0} \\ {1} \\ {0} \\ {0} \\ {0}\end{array}\right]
	\]
	
	请确定$ \Vx  $在子空间$ \SetU $上的正交投影$ \pi_{\SetU}(\boldsymbol{x}) ;$
	
\end{exercise}
\begin{Solution}
首先,将$U$中的基向量构成矩阵$A$:
\[
A=\left[\begin{array}{ll}
	0 & 1 \\
	-1 & -1 \\
	2 & 1 \\
	0 & -1 \\
	1 & 1 
\end{array}\right]
\]

易计算出
\[
A\T A=\left[\begin{array}{ll}
	6 & 4 \\
	4 & 5 
\end{array}\right], \quad
(A\T A)^{-1}=\left[\begin{array}{ll}
	5/14 & -2/7 \\
	-2/7 & 3/7 
\end{array}\right]
\]

\hfill \textcolor{red}{\textbf{(4分)}}
 
然后,计算投影矩阵$P_U$:
\[
P_U=\left[\begin{array}{ccccc}
	3/7 & -1/7 & -1/7 & -3/7 & 1/7 \\
	-1/7 & 3/14 & -2/7 & 1/7 & -3/14 \\
	-1/7 & -2/7 & 5/7 & 1/7 & 2/7 \\
	-3/7 & 1/7 & 1/7 & 3/7 & -1/7 \\
	1/7 & -3/14 & 2/7 & -1/7 & 3/14 
\end{array}\right]
\]
因此,投影
\[
\pi_U(x) = P_U x = (-1/7, ~ 3/14, ~ -2/7, ~ 1/7, ~ -3/14)
\]

\hfill \textcolor{red}{\textbf{(6分)}}
	
\end{Solution}




\begin{exercise}(13')
	完成下列矩阵函数求梯度:
	\begin{itemize}
		\item [(1)] 求函数$f(\Vx)=\Va\T \Vx$ 和 $g(\Vx)=\Vx\T \MA \Vx$ 的 Hessian 矩阵;
		\item [(2)] 考虑一个两层的全连接神经网络:
		\[
		\Vy=\Vf(\Vx)=\mathrm{ReLU}(\MA_2(\mathrm{ReLU}(\MA_1\Vx+\Vb_1))+\Vb_2)	
		\]
		其中\[
		\MA_1=\begin{bmatrix}
			2&3\\
			-2&1\\
			3&-1
		\end{bmatrix}	,\MA_2=\begin{bmatrix}
			3&-1&0\\
			1&-2&2
		\end{bmatrix},\Vb_1=\begin{bmatrix}
			-1\\4\\-2
		\end{bmatrix},\Vb_2=\begin{bmatrix}
			2\\-3
		\end{bmatrix}
		\]
		假设输入为$\Vx=(1,-1)$,并且对应的真实输出为$\hat{\Vy}=(0,1)$,采用平方损失$L=\frac12\|\Vy-\hat{\Vy}\|_2^2$。
		试计算函数$L$关于$\Vb_1$的梯度。
	\end{itemize}
\end{exercise}

\begin{Solution}
	(1)因为$\frac{\partial f}{\partial \Vx} = \Va $,因此 $H_f = \MO$。又
	$$
	\begin{aligned}
		d(g)&= Tr( d\Vx\T \cdot \MA \Vx + \MA \Vx d\Vx)\\
		&=Tr(\Vx\T (\MA\T + \MA) d\Vx)
	\end{aligned}
	$$
	所以 Hessian 矩阵$H_g = \MA\T + \MA$。
	\hfill \textcolor{red}{\textbf{(5分)}}
	
	(2)
	
	先计算前项过程:\[
	\MA_1\Vx+\Vb_1=\begin{bmatrix}
		-2\\1\\2
	\end{bmatrix},	\MA_2(\mathrm{ReLU}(\MA_1\Vx+\Vb_1))+\Vb_2=\begin{bmatrix}
		1\\0
	\end{bmatrix}
	\]
	故$\Vy=(1,0)$,
	从而$L=1$。
	记\[
	\Vk=\mathrm{ReLU}(\MA_1\Vx+\Vb_1)
	\]然后分别计算\[
	\frac{\partial L}{\partial \Vy}=\begin{bmatrix}
		1\\-1
	\end{bmatrix},	\frac{\partial \Vy\T}{\partial \Vk}=\begin{bmatrix}
		3&0\\-1&0 \\0&0
	\end{bmatrix},	\frac{\partial \Vk\T}{\partial \Vb_1}=\begin{bmatrix}
		0&0&0\\0&1&0\\0&0&1
	\end{bmatrix}	
	\]
	
	所以有\[
	\frac{\partial L}{\partial \Vb_1}=\frac{\partial \Vk\T}{\partial \Vb_1}\frac{\partial \Vy\T}{\partial \Vk}\frac{\partial L}{\partial \Vy}	=\begin{bmatrix}
		0&0&0\\0&1&0\\0&0&1
	\end{bmatrix}	\begin{bmatrix}
		3&0\\-1&0\\0&0
	\end{bmatrix}\begin{bmatrix}
		1\\-1
	\end{bmatrix}=\begin{bmatrix}
		0\\-1\\0
	\end{bmatrix}
	\]
	\hfill \textcolor{red}{\textbf{(8分)}}
	
\end{Solution}


\begin{exercise}(13')
	完成下列问题:
 \begin{itemize}
 	\item [(1)] 设已知矩阵$ \MM \in \mathbb{R}^{m \times n} $的SVD分解为 $\MU \MSigma \MV\T$,请写出拼接矩阵 $\MA = [\MM \quad \MM]$ 的奇异值分解;
 	\item [(2)] 令 $\MA$ 为 $m\times n$矩阵,且 $\MP$ 为 $m\times m$正交矩阵。证明:$\MP \MA$与$\MA$的奇异值相同。并说明矩阵$\MP \MA$与$\MA$的左、右奇异向量有何关系?
 \end{itemize}
\end{exercise}
\begin{Solution}
	(1)因为$\MM=\MU \MSigma \MV\T$,因此
		\[
		\MA \MA\T = 2 \MM \MM\T = \MU \cdot 2\MSigma_{m \times m}^2 \cdot \MU\T.
		\]
		故
		$\MA$的奇异值为 $\sqrt{2}\MSigma_{m \times 2n}$,左奇异向量构成的正交矩阵就是$\MU$。
		\hfill \textcolor{red}{\textbf{(3分)}}
		
		由$\MU\T \MA$与右奇异向量(设为$\MV_{\MA}$)的关系可直接计算出
		$$[\MSigma_{m \times n}\MV\T \quad \MSigma_{m \times n}\MV\T] = \sqrt{2} \MSigma_{m \times 2n} \MV_{\MA}\T$$
		对比等式两边可得 $\MV_{\MA}\T$ 的前 $n$ 行即为 $[\frac{1}{\sqrt{2}}\MV\T \quad \frac{1}{\sqrt{2}}\MV\T]$。对此进行正交扩充,可得
		$$
		\MV_{\MA}\T = 
		\left[\begin{array}{ll}
			\frac{1}{\sqrt{2}}\MV\T & \frac{1}{\sqrt{2}}\MV\T \\
			\frac{1}{\sqrt{2}}\MV\T & -\frac{1}{\sqrt{2}}\MV\T
		\end{array}\right]
		$$
\hfill \textcolor{red}{\textbf{(4分)}}

	(2)因为 $(\MP\MA)\T (\MP\MA) = \MA\T\MP\T\MP\MA=\MA\T\MA$,故它们的奇异值相同。
	
	\hfill \textcolor{red}{\textbf{(2分)}}
	
	假设$\MA$的SVD分解为
	$$
	\MA=U \Sigma V^{\mathrm{T}}
	$$
	其中 $U, V \in \mathbb{R}^{n \times n}$ 正交。故
	$$
	\MP\MA=\MP U \Sigma V^{\mathrm{T}}
	$$
	因为$\MP$为正交矩阵,因此$\MP U$为其左奇异向量,$V$就是$\MP \MA$的右奇异向量。
	
	\hfill \textcolor{red}{\textbf{(4分)}}	
	
\end{Solution}


\begin{exercise}(12')
	已知矩阵\[ \MM = \begin{pmatrix}
		5&1&2\\
		-2&-3&4\\
		1&3&-1\\
	\end{pmatrix}
	\]
	\begin{itemize}
		\item [(1)] 计算矩阵 $\MM$ 的 $1$ 范数和$\infty$范数;
		\item [(2)] 证明:如果矩阵 $\MA = [\Va_1, \Va_2, \cdots, \Va_n]$ 是按列分块的,那么 $\|\MA\|_F^2 = \|\Va_1\|_2^2 + \|\Va_2\|_2^2 + \cdots + \|\Va_n\|_2^2$。
		\item [(3)] 设$ a_1,a_2,\cdots,a_n $是$ n $个正数,证明:由
		$$ \Omega(\Vx)=\left(\sum_{i=1}^na_ix_i^2\right)^{\frac{1}{2}} $$
		定义的函数$ \Omega: \NR^n\to\NR $是一个范数。
	\end{itemize}
\end{exercise}
\begin{Solution}
	(1)	
	$$\|\MM\|_1=\max\{5+|-2|+1, 1+|-3|+3, 2+4+|-1|\}=8$$ 
	$$\|\MM\|_{\infty}=\max\{5+1+2, |-2|+|-3|+4, 1+3+|-1|\}=9$$
	\hfill \textcolor{red}{\textbf{(2分)}}
	
	(2)因为矩阵 $\MA$ 的 $F$范数的平方是矩阵各元素的平方和,而$\Va_1$的$\ell_2$范数的平方是矩阵第一列元素的平方和,以此类推,$\Va_n$的$\ell_2$范数的平方是矩阵第$n$列元素的平方和。因此,的证
	\hfill \textcolor{red}{\textbf{(5分)}}
	
	(3)可以通过定义证明该函数具有范数的非负性、齐次性和三角不等式,只要证明过程合理即可。 \hfill \textcolor{red}{\textbf{(5分)}}
\end{Solution}


\begin{exercise}(12')
	求解以下问题:
\begin{itemize}
	\item [(1)] 假设$X_{1} \rightarrow X_{2} \rightarrow X_{3} \rightarrow \cdots \rightarrow X_{n}$ 是一个马尔科夫链,即
	$$
	p\left(x_{1}, x_{2}, \ldots, x_{n}\right)=p\left(x_{1}\right) p\left(x_{2} \mid x_{1}\right) \cdots p\left(x_{n} \mid x_{n-1}\right)
	$$
	试化简 $I\left(X_{1} ; X_{2}, \ldots, X_{n}\right)$。
	\item [(2)] $X$和$Y$是$\{0,1,2,3\}$上的独立、等概率分布的随机变量,求 $H(X+Y)$。
\end{itemize}

\end{exercise}
\begin{Solution}
	(1)$$
	\begin{aligned}
		I\left(X_{1} ; X_{2}, \ldots, X_{n}\right) &=H\left(X_{1}\right)-H\left(X_{1} \mid X_{2}, \ldots, X_{n}\right) \\
		&=H\left(X_{1}\right)-\left[H\left(X_{1}, X_{2}, \ldots, X_{n}\right)-H\left(X_{2}, \ldots, X_{n}\right)\right] \\
		&=H\left(X_{1}\right)-\left[\sum_{i=1}^{n} H\left(X_{i} \mid X_{i-1}, \ldots, X_{1}\right)-\sum_{i=2}^{n} H\left(X_{i} \mid X_{i-1}, \ldots, X_{2}\right)\right] \\
		&=H\left(X_{1}\right)-\left[\left(H\left(X_{1}\right)+\sum_{i=2}^{n} H\left(X_{i} \mid X_{i-1}\right)\right) -\left(H\left(X_{2}\right)+\sum_{i=3}^{n} H\left(X_{i} \mid X_{i-1}\right)\right)\right] \\
		&=H\left(X_{2}\right)-H\left(X_{2} \mid X_{1}\right) \\
		&=I\left(X_{2} ; X_{1}\right) \\
		&=I\left(X_{1} ; X_{2}\right)
	\end{aligned}
	$$
	\hfill \textcolor{red}{\textbf{(7分)}}
	
	
	
	(2)。
	解:$X+Y$的分布律
	
	\begin{table}[H]
		\centering
		\begin{tabular}{cccccccc}
			\hline
			$X+Y$ & 0 & 1 & 2 & 3 & 4 & 5 & 6 \\
			\hline
			概率 & 1/16 & 2/16 & 3/16 & 4/16 & 3/16 & 2/16 & 1/16 \\
			\hline
		\end{tabular}
	\end{table}
	
	因此
	\[
	H(X+Y) =-( 2\frac{1}{16} \log_2 \frac{1}{16} + 2\frac{2}{16} \log_2 \frac{2}{16} + 2\frac{3}{16} \log_2 \frac{3}{16} + \frac{4}{16} \log_2 \frac{4}{16})
	\]
	也可以用其他对数,这里可化简也可以不化简。
	
	\hfill \textcolor{red}{\textbf{(5分)}}
\end{Solution}




\begin{exercise}(13')
	求解下述问题:
	\begin{itemize}
		\item[(1)] 请计算负熵函数 $f(\Vx)=\sum_{i=1}^{n} x_i \log x_i$ 的共轭函数。
		
		
		\item [(2)] 用Lagrange乘子法求欠定方程$ \MA\Vx=\Vb $的最小二范数解,其中$ \MA\in\NR^{m\times n},m\leq n,\rank(\MA)=m $。
	\end{itemize}
\end{exercise}
\begin{Solution}
	(1)设 $g(x) = \langle x,y \rangle-f(x) = \langle x,y \rangle-\sum_{i=1}^{n} x_i \log x_i$,
	对$x$求梯度可得:
	\[
	\nabla g = [y_1-\log x_1 - 1~ y_2-\log x_2 - 1~ \cdots~ y_n-\log x_n - 1]\T
	\]
	因此,可得最优解为 $x_i = e^{y_i - 1}$,代入即得:
	\[
	f^* (\Vy) = \sum_{i=1}^{n} e^{y_i - 1}.
	\]
	
	\hfill \textcolor{red}{\textbf{(6分)}}
	
	(2)优化问题为\[ 	\begin{aligned}
		\texttt{min} \quad & f(\Vx)=\frac{1}{2}\Vert \Vx\Vert_{2}^2 \\
		\texttt{suject to} \quad & \MA\Vx=\Vb
	\end{aligned} \]
	Lagrange函数为:
	\[ L(\Vx,
	\Vlambda)=\frac{1}{2}\Vx\T\Vx-\Vlambda\T(\MA\Vx-\Vb) \]
	\[ \frac{\partial L}{\partial \Vx}= \Vx -\MA\T\Vlambda\]
	令$ \frac{\partial L}{\partial \Vx}=0 $,有:
	\[ \Vx=\MA\T\Vlambda \]
	\[ g(\Vlambda)=-\frac{1}{2}\Vlambda\T\MA\MA\T\lambda+\Vlambda\T\Vb \]
	令$ \frac{\partial g}{\partial \Vlambda}=0  $:
	\[ -\MA\MA\T\lambda+\Vb=0 \]
	由$ \MA\in\NR^{m\times n},\rank(\MA)=m $得$ \MA\MA\T $可逆,因此
	\[ \Vlambda=(\MA\MA\T)^{-1}\Vb \]
	因此,$ \Vx $满足$ \MA\Vx=\Vb $的最小二范数解: \[ \Vx=\MA\T(\MA\MA\T)^{-1}\Vb  \]
	
	\hfill \textcolor{red}{\textbf{(7分)}}
\end{Solution}

\begin{exercise}(14')
	求解如下优化问题
	\begin{itemize}
		\item [(1)] 使用梯度下降法和固定步长$\lambda =0.1$计算$\min f(x)=(x_{1}-1)^{2}+16(x_{2}-2)^{2} \quad $, 初始点$x^{(0)}=(3,2)^{T}$,迭代至第三步后终止,请计算迭代过程中的解、梯度和函数值(第三步不用计算梯度和函数值,计算结果保留小数点后两位)。
		\item[(2)] 请谈谈牛顿法与BFGS方法之间的区别与联系。
	\end{itemize}
\end{exercise}

\begin{Solution}
	(1)易求得梯度为:
	\[
	\nabla f = [2(x_1 - 1) \quad 32(x_2 - 2) ]\T
	\]
	具体迭代结果:\\
	\begin{center}
		$\begin{array}{c|c|c|c}
			\hline k & x^{(k)^{\mathrm{T}}} &g_k^{\mathrm{T}}& f_k \\
			\hline 0 & (3,2) & (4,0) & 4  \\
			1 & (2.6,2) &(3.2, 0) & 2.56 \\
			2 & (2.28,2) & (2.56, 0) & 1.6384 \\
			3 & (2.024,2) &  &  \\
			\hline
		\end{array}$
	\end{center}

	\hfill \textcolor{red}{\textbf{(8分)}}
	
	(2)牛顿法(Newton's method)和BFGS算法(Broyden-Fletcher-Goldfarb-Shanno algorithm)都是优化算法,用于求解无约束非线性优化问题。它们都属于迭代优化算法,但有一些关键的区别。
	
	\textbf{1. 原理和基本思想:}
	\begin{itemize}
		\item 牛顿法:基于二阶导数(Hessian矩阵)的信息,通过迭代更新当前点的位置,以找到目标函数的极小值点。牛顿法通常收敛速度很快,但需要计算和存储Hessian矩阵,这可能在高维问题中成为一个挑战。
		
		\item BFGS算法:也是基于二阶导数的信息,但它通过估计Hessian矩阵的逆矩阵来进行迭代。相较于牛顿法,BFGS不需要直接计算和存储Hessian矩阵,这降低了存储和计算的复杂性。
	\end{itemize}
	
	\textbf{2. 更新规则:}
	\begin{itemize}
		\item 牛顿法:通过解线性方程系统来计算下降方向。牛顿法的更新规则是直接基于Hessian矩阵和梯度的。
		
		\item BFGS算法:通过迭代更新逆Hessian矩阵的估计值,以获得下降方向。BFGS的更新规则更为复杂,但避免了直接计算Hessian矩阵的需求。
	\end{itemize}
	
	\textbf{3. 收敛性:}
	\begin{itemize}
		\item 牛顿法:在适当的条件下,牛顿法通常具有二次收敛性,收敛速度较快,但可能受到Hessian矩阵的病态性(ill-conditioning)的影响。
		
		\item BFGS算法:通常也具有二次收敛性,但相对于牛顿法更具有稳定性,特别是在高维问题中,因为它不需要直接处理Hessian矩阵。
	\end{itemize}
	
	总体上,只要言之有理,且答到关键点即可给分。
	
	\hfill \textcolor{red}{\textbf{(6分)}}
\end{Solution}




\end{document}

