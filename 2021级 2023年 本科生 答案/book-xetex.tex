\documentclass[12pt,a4paper,openany,twoside]{ctexbook}
%-------------------------------------宏包引用---------------------------------------------------
\usepackage[paperwidth=185mm,paperheight=230mm,textheight=190mm,textwidth=145mm,left=20mm,right=20mm, top=25mm, bottom=15mm]{geometry}            %定义版面
%--------------------------------------------------------------------------------------
\usepackage{fontspec}
\usepackage{xunicode}
\usepackage{xltxtra}
%--------------------------------------------------------------------------------------
\usepackage[listings,theorems]{tcolorbox}
\usepackage{fancybox}                 % 边框,有阴影,fancybox提供了五种式样\fbox,\shadowbox,\doublebox,\ovalbox,\Ovalbox。
\usepackage{colortbl}                   % 单元格加背景
\usepackage{fancyhdr}                 % 页眉和页脚的相关定义
\usepackage[CJKbookmarks, colorlinks, bookmarksnumbered=true,pdfstartview=FitH,linkcolor=black]{hyperref}   % 书签功能,选项去掉链接红色方框
\usepackage{mathtools}
\usepackage{amssymb}
\usepackage{bm}	% 处理数学公式中的黑斜体的宏包

\usepackage{subfigure}

\usepackage{algorithm}
\usepackage{algorithmic}

\usepackage[amsmath,thmmarks]{ntheorem}         % 定理类环境宏包,其中 amsmath 选项用来兼容 AMS LaTeX 的宏包thmmarks 选项,可以自动恰当地放置定理类环境的结束标记;它还能像图形目录那样生成定理类环境目录

\usepackage{fancyref}                                % 支持引用的宏包   %引用宏包所在位置
%-----------------------------------------------------主文档 格式定义---------------------------------
\addtolength{\headsep}{-0.1cm}        %页眉位置
%\addtolength{\footskip}{0.4cm}       %页脚位置
%-----------------------------------------------------设定字体等------------------------------
\setmainfont{Times New Roman}    % 缺省字体
%\setCJKfamilyfont{song}{SimSun}
%\setCJKfamilyfont{hei}{SimHei}
%\setCJKfamilyfont{kai}{KaiTi}
%\setCJKfamilyfont{fs}{FangSong}
%\setCJKfamilyfont{li}{LiSu}
%\setCJKfamilyfont{you}{YouYuan}
%\setCJKfamilyfont{yahei}{Microsoft YaHei}
%\setCJKfamilyfont{xingkai}{STXingkai}
%\setCJKfamilyfont{xinwei}{STXinwei}
%\setCJKfamilyfont{fzyao}{FZYaoTi}
%\setCJKfamilyfont{fzshu}{FZShuTi}
%-------------------------------------------------------------------
\newCJKfontfamily\song{SimSun}
\newCJKfontfamily\hei{SimHei}
\newCJKfontfamily\kai{KaiTi}
\newCJKfontfamily\fs{FangSong}
\newCJKfontfamily\li{LiSu}
\newCJKfontfamily\you{YouYuan}
%\newCJKfontfamily\yahei{Microsoft YaHei}
\newCJKfontfamily\xingkai{STXingkai}
\newCJKfontfamily\xinwei{STXinwei}
\newCJKfontfamily\fzyao{FZYaoTi}
\newCJKfontfamily\fzshu{FZShuTi}
%------------------------------------------------------------------------------------------------------
\newcommand{\linghao}{\fontsize{48pt}{60pt}\selectfont}      % 零号, 1.4倍行距
\newcommand{\chuhao}{\fontsize{42pt}{\baselineskip}\selectfont}  % 初号& 42pt
\newcommand{\xiaochuhao}{\fontsize{36pt}{\baselineskip}\selectfont} % 小初号& 36pt
\newcommand{\yihao}{\fontsize{28pt}{38pt}\selectfont}        % 一号, 1.4倍行距
\newcommand{\erhao}{\fontsize{21pt}{28pt}\selectfont}        % 二号, 1.25倍行距
\newcommand{\xiaoer}{\fontsize{18pt}{18pt}\selectfont}       % 小二, 单倍行距
\newcommand{\sanhao}{\fontsize{16pt}{24pt}\selectfont}       % 三号, 1.5倍行距
\newcommand{\xiaosan}{\fontsize{15pt}{22pt}\selectfont}      % 小三, 1.5倍行距
\newcommand{\sihao}{\fontsize{14pt}{21pt}\selectfont}        % 四号, 1.5倍行距
\newcommand{\xiaosihao}{\fontsize{12pt}{18pt}\selectfont}    % 小四, 1.5倍行距
\newcommand{\dawuhao}{\fontsize{11pt}{11pt}\selectfont}      % 大五号, 单倍行距
\newcommand{\wuhao}{\fontsize{10.5pt}{10.5pt}\selectfont}    % 五号, 单倍行距
\newcommand{\xiaowuhao}{\fontsize{10pt}{10pt}\selectfont}    % 小五号, 单倍行距
\newcommand{\liuhao}{\fontsize{7.875pt}{7.875pt}\selectfont} % 六号, 单倍行距
\newcommand{\qihao}{\fontsize{5.25pt}{\baselineskip}\selectfont}
%--------------------------------------------------------------------------------------------------------
%-----------------------------------------------------------定义颜色---------------
\definecolor{blueblack}{cmyk}{0,0,0,0.35}%浅黑
\definecolor{darkblue}{cmyk}{1,0,0,0}%纯蓝
\definecolor{lightblue}{cmyk}{0.15,0,0,0}%浅蓝
%--------------------------------------------------------设定标题颜色--------------
\CTEXsetup[format+={\li \color{black}}]{chapter}
\CTEXsetup[format+={\li \color{black}}]{section}
\CTEXsetup[format+={\li \color{black}}]{subsection}

%-----------重定义带编号标题----------------------------
\theoremstyle{plain}
\theoremheaderfont{\hei\rmfamily}
\theorembodyfont{\song\rmfamily}
\theoremseparator{~}%帽号 : 改为 ~ 变空格
\newtheorem{block}{\indent\color{darkblue}}
\newtheorem{definition}{\indent\color{darkblue}定义}[section]
\newtheorem{proposition}{\indent\color{darkblue}命题}[section]
\newtheorem{theorem}{\indent\color{darkblue}定理}[section]
\newtheorem{lemma}{\indent\color{darkblue}引理}[section]
\newtheorem{axiom}{\indent\color{darkblue}公理}[section]
\newtheorem{corollary}{\indent\color{darkblue}推论}[section]
\newtheorem{character}{\indent\color{darkblue}性质}[section]


\newtheorem{exam}{\indent\hei \color{darkblue}例}[chapter]
\newtheorem{exercise}{\indent\hei \color{darkblue}题}
\newtheorem{example}{\hspace{2em}\hei\color{white}Matlab程序}[section]
\newcommand{\program}[1]{\noindent\parbox{14.6cm}{\includegraphics[width=3.1cm]{fig/sjwt.jpg}
		\vspace{-1.12cm}\example\textcolor{darkblue}{\hei#1}}\vspace{0.0cm}}  %实际问题标志



%------------------------------------------重定义不带编号标题----------------------------
\theoremstyle{nonumberplain} \theoremseparator{~}    %帽号 : 改为 ~ 变空格
%\theoremsymbol{$\blacksquare$}                      %在证明结束行尾加上一个黑色方块
\newtheorem{proof}{\indent\hei \textcolor{darkblue}{证明}}
\newtheorem{Solution}{\indent\hei \textcolor{darkblue}{解}}
\newtheorem{REFERENCE}{\indent\hei\sihao \textcolor{darkblue}{参考文献}}
%==========程序框====================================
\newcommand{\programbox}[2]{
	\begin{tcolorbox}[colback=white,colframe=darkblue,title=\example #1]
		#2
\end{tcolorbox}}
%==========程序框结束====================================

%==========命题框====================================
\newcommand{\propositionbox}[1]{\vspace{0.3cm}
	\begin{colorboxed}[boxcolor=darkblue,bgcolor=white]
		\vspace{-0.3cm}\proposition #1
\end{colorboxed}}
%==========命题框结束====================================

%==========引理框====================================
\newcommand{\lemmabox}[1]{\vspace{0.3cm}
	\begin{colorboxed}[boxcolor=darkblue,bgcolor=white]
		\vspace{-0.3cm}\lemma #1
\end{colorboxed}}
%==========引理框结束====================================

%==========定理框====================================
\newcommand{\theorebox}[1]{\vspace{0.3cm}
	\begin{colorboxed}[boxcolor=darkblue,bgcolor=lightblue]
		\vspace{-0.3cm}\theorem #1
\end{colorboxed}}
%==========定理框结束====================================

%==========推论框====================================
\newcommand{\corolbox}[1]{\vspace{0.3cm}
	\begin{colorboxed}[boxcolor=darkblue,bgcolor=white]
		\vspace{-0.3cm}\corollary #1
\end{colorboxed}}
%==========推论框结束====================================

%==========性质框====================================
\newcommand{\charactbox}[1]{\vspace{0.3cm}
	\begin{colorboxed}[boxcolor=darkblue,bgcolor=white]
		\vspace{-0.3cm}\character #1
\end{colorboxed}}
%==========性质框结束====================================

%==========定义框====================================
\newcommand{\definibox}[2]{\vspace{0.3cm}
	\begin{colorboxed}[boxcolor=darkblue,fgcolor=white,bgcolor=darkblue]
		\vspace{-0.4cm}
		\definition\hei\color{white}#1
		\vspace{0.04cm}
	\end{colorboxed}
	\vspace{-0.7cm} \setlength{\fboxsep}{4pt}
	\begin{colorboxed}[boxcolor=darkblue,fgcolor=black,bgcolor=lightblue]
		#2
	\end{colorboxed}\color{black}}
%==========定义框结束====================================

%%-----------------------------------------------------------定义、定理环境-------------------------
%\newcounter{myDefinition}[chapter]\def\themyDefinition{\thechapter.\arabic{myDefinition}}
%\newcounter{myTheorem}[chapter]\def\themyTheorem{\thechapter.\arabic{myTheorem}}
%\newcounter{myCorollary}[chapter]\def\themyCorollary{\thechapter.\arabic{myCorollary}}
%
%\tcbmaketheorem{defi}{定义}{fonttitle=\bfseries\upshape, fontupper=\slshape, arc=0mm, colback=lightblue,colframe=darkblue}{myDefinition}{Definition}
%\tcbmaketheorem{theo}{定理}{fonttitle=\bfseries\upshape, fontupper=\slshape, arc=0mm, colback=lightblue,colframe=darkblue}{myTheorem}{Theorem}
%\tcbmaketheorem{coro}{推论}{fonttitle=\bfseries\upshape, fontupper=\slshape, arc=0mm, colback=lightblue,colframe=darkblue}{myCorollary}{Corollary}
%%------------------------------------------------------------------------------
%\newtheorem{proof}{\indent\hei \textcolor{darkblue}{证明}}
%\newtheorem{Solution}{\indent\hei \textcolor{darkblue}{解}}
%------------------------------------------------定义页眉下单隔线----------------
\newcommand{\makeheadrule}{\makebox[0pt][l]{\color{darkblue}\rule[.7\baselineskip]{\headwidth}{0.3pt}}\vskip-.8\baselineskip}
%-----------------------------------------------定义页眉下双隔线----------------
\makeatletter
\renewcommand{\headrule}{{\if@fancyplain\let\headrulewidth\plainheadrulewidth\fi\makeheadrule}}
\pagestyle{fancy}
\renewcommand{\chaptermark}[1]{\markboth{第\chaptername 章\quad #1}{}}    %去掉章标题中的数字
\renewcommand{\sectionmark}[1]{\markright{\thesection\quad #1}{}}    %去掉节标题中的点
\fancyhf{} %清空页眉
\fancyhead[RO]{\kai{\footnotesize.~\color{darkblue}\thepage~.}}         % 奇数页码显示左边
\fancyhead[LE]{\kai{\footnotesize.~\color{darkblue}\thepage~.}}         % 偶数页码显示右边
\fancyhead[CO]{\song\footnotesize\color{darkblue}\rightmark} % 奇数页码中间显示节标题
\fancyhead[CE]{\song\footnotesize\color{darkblue}\leftmark}  % 偶数页码中间显示章标题
%---------------------------------------------------------------------------------------------------------------------


    %格式所在位置
% !Mode:: "TeX:UTF-8"
\newcommand{\normmm}[1]{{\left\vert\kern-0.25ex\left\vert\kern-0.25ex\left\vert #1 
		\right\vert\kern-0.25ex\right\vert\kern-0.25ex\right\vert}}

\newcommand{\argmax}{\arg\max}
\newcommand{\argmin}{\arg\min}
\newcommand{\sigmoid}{\text{sigmoid}}
\newcommand{\norm}[1]{\left\lVert#1\right\rVert}
\newcommand{\inprod}[2]{\langle#1,#2\rangle}
\newcommand{\Tr}{\text{Tr}}
\renewcommand{\d}{\text{d}}

\newcommand{\Var}{\text{Var}}
\newcommand{\Cov}{\text{Cov}}
\newcommand{\plim}{\text{plim}}
\newcommand{\Tsp}{\top}

\newcommand{\diag}{\mathop{\rm diag}\nolimits}
\newcommand{\rank}{\mathop{\rm rank}\nolimits}
\newcommand{\Laplace}{\mathop{\rm Laplace}\nolimits}
\newcommand{\Bernoulli}{\mathop{\rm Bernoulli}\nolimits}
\newcommand{\Uniform}{\mathop{\rm Uniform}\nolimits}
\newcommand{\Possion}{\mathop{\rm Possion}\nolimits}
\newcommand{\Ber}{\mathop{\rm Ber}\nolimits}
\newcommand{\Bin}{\mathop{\rm Bin}\nolimits}
\newcommand{\Cat}{\mathop{\rm Cat}\nolimits}
\newcommand{\Dir}{\mathop{\rm Dir}\nolimits}
\newcommand{\Lap}{\mathop{\rm Lap}\nolimits}
\newcommand{\sigm}{\mathop{\rm sigm}\nolimits}

\newcommand{\RSS}{\mathop{\rm RSS}\nolimits}
\newcommand{\se}{\mathop{\rm se}\nolimits}
\newcommand{\logit}{\mathop{\rm logit}\nolimits}
\newcommand{\dom}{\mathop{\bf dom}\nolimits}
\newcommand{\epi}{\mathop{\bf epi}\nolimits}
\newcommand{\hypo}{\mathop{\bf hypo}\nolimits}
\newcommand{\prob}{\mathop{\bf prob}\nolimits}
\newcommand{\relint}{\mathop{\bf relint}\nolimits}
\newcommand{\aff}{\mathop{\bf aff}\nolimits}
\newcommand{\spn}{\mathop{\bf span}\nolimits}
% Scala 标量 暂不使用
\newcommand{\Sa}{\mathit{a}}
\newcommand{\Sb}{\mathit{b}}
\newcommand{\Sc}{\mathit{c}}
\newcommand{\Sd}{\mathit{d}}
\newcommand{\Se}{\mathit{e}}
\newcommand{\Sf}{\mathit{f}}
\newcommand{\Sg}{\mathit{g}}
\newcommand{\Sh}{\mathit{h}}
\newcommand{\Si}{\mathit{i}}
\newcommand{\Sj}{\mathit{j}}
\newcommand{\Sk}{\mathit{k}}
\newcommand{\Sl}{\mathit{l}}
\newcommand{\Sm}{\mathit{m}}
\newcommand{\Sn}{\mathit{n}}
\newcommand{\So}{\mathit{o}}
\newcommand{\Sp}{\mathit{p}}
\newcommand{\Sq}{\mathit{q}}
\newcommand{\Sr}{\mathit{r}}
\newcommand{\Ss}{\mathit{s}}
\newcommand{\St}{\mathit{t}}
\newcommand{\Su}{\mathit{u}}
\newcommand{\Sv}{\mathit{v}}
\newcommand{\Sw}{\mathit{w}}
\newcommand{\Sx}{\mathit{x}}
\newcommand{\Sy}{\mathit{y}}
\newcommand{\Sz}{\mathit{z}}

\newcommand{\SA}{\mathit{A}}
\newcommand{\SB}{\mathit{B}}
\newcommand{\SC}{\mathit{C}}
\newcommand{\SD}{\mathit{D}}
\newcommand{\SE}{\mathit{E}}
\newcommand{\SF}{\mathit{F}}
\newcommand{\SG}{\mathit{G}}
\newcommand{\SH}{\mathit{H}}
\newcommand{\SJ}{\mathit{J}}
\newcommand{\SK}{\mathit{K}}
\newcommand{\SI}{\mathit{L}}
\newcommand{\SM}{\mathit{M}}
\newcommand{\SN}{\mathit{N}}
\newcommand{\SO}{\mathit{O}}
\newcommand{\SP}{\mathit{P}}
\newcommand{\SQ}{\mathit{Q}}
\newcommand{\SR}{\mathit{R}}
\newcommand{\ST}{\mathit{T}}
\newcommand{\SU}{\mathit{U}}
\newcommand{\SV}{\mathit{V}}
\newcommand{\SW}{\mathit{W}}
\newcommand{\SX}{\mathit{X}}
\newcommand{\SY}{\mathit{Y}}
\newcommand{\SZ}{\mathit{Z}}



% Vector 向量 \VI 全1向量 \VO 零向量
\newcommand{\Va}{\boldsymbol{\mathit{a}}}
\newcommand{\Vb}{\boldsymbol{\mathit{b}}}
\newcommand{\Vc}{\boldsymbol{\mathit{c}}}
\newcommand{\Vd}{\boldsymbol{\mathit{d}}}
\newcommand{\Ve}{\boldsymbol{\mathit{e}}}
\newcommand{\Vf}{\boldsymbol{\mathit{f}}}
\newcommand{\Vg}{\boldsymbol{\mathit{g}}}
\newcommand{\Vh}{\boldsymbol{\mathit{h}}}
\newcommand{\Vi}{\boldsymbol{\mathit{i}}}
\newcommand{\Vj}{\boldsymbol{\mathit{j}}}
\newcommand{\Vk}{\boldsymbol{\mathit{k}}}
\newcommand{\Vl}{\boldsymbol{\mathit{l}}}
\newcommand{\Vm}{\boldsymbol{\mathit{m}}}
\newcommand{\Vn}{\boldsymbol{\mathit{n}}}
\newcommand{\Vo}{\boldsymbol{\mathit{o}}}
\newcommand{\Vp}{\boldsymbol{\mathit{p}}}
\newcommand{\Vq}{\boldsymbol{\mathit{q}}}
\newcommand{\Vr}{\boldsymbol{\mathit{r}}}
\newcommand{\Vs}{\boldsymbol{\mathit{s}}}
\newcommand{\Vt}{\boldsymbol{\mathit{t}}}
\newcommand{\Vu}{\boldsymbol{\mathit{u}}}
\newcommand{\Vv}{\boldsymbol{\mathit{v}}}
\newcommand{\Vw}{\boldsymbol{\mathit{w}}}
\newcommand{\Vx}{\boldsymbol{\mathit{x}}}
\newcommand{\Vy}{\boldsymbol{\mathit{y}}}
\newcommand{\Vz}{\boldsymbol{\mathit{z}}}

%零向量,由于命令中不能出现数字,用字母“O”代替数字“0”
\newcommand{\VO}{\boldsymbol{0}}
\newcommand{\VI}{\boldsymbol{1}}

% Matrix 矩阵
\newcommand{\MA}{\boldsymbol{\mathit{A}}}
\newcommand{\MB}{\boldsymbol{\mathit{B}}}
\newcommand{\MC}{\boldsymbol{\mathit{C}}}
\newcommand{\MD}{\boldsymbol{\mathit{D}}}
\newcommand{\ME}{\boldsymbol{\mathit{E}}}
\newcommand{\MF}{\boldsymbol{\mathit{F}}}
\newcommand{\MG}{\boldsymbol{\mathit{G}}}
\newcommand{\MH}{\boldsymbol{\mathit{H}}}
\newcommand{\MI}{\boldsymbol{\mathit{I}}}
\newcommand{\MJ}{\boldsymbol{\mathit{J}}}
\newcommand{\MK}{\boldsymbol{\mathit{K}}}
\newcommand{\ML}{\boldsymbol{\mathit{L}}}
\newcommand{\MM}{\boldsymbol{\mathit{M}}}
\newcommand{\MN}{\boldsymbol{\mathit{N}}}
\newcommand{\MO}{\boldsymbol{\mathit{O}}}
\newcommand{\MP}{\boldsymbol{\mathit{P}}}
\newcommand{\MQ}{\boldsymbol{\mathit{Q}}}
\newcommand{\MR}{\boldsymbol{\mathit{R}}}
\newcommand{\MS}{\boldsymbol{\mathit{S}}}
\newcommand{\MT}{\boldsymbol{\mathit{T}}}
\newcommand{\MU}{\boldsymbol{\mathit{U}}}
\newcommand{\MV}{\boldsymbol{\mathit{V}}}
\newcommand{\MW}{\boldsymbol{\mathit{W}}}
\newcommand{\MX}{\boldsymbol{\mathit{X}}}
\newcommand{\MY}{\boldsymbol{\mathit{Y}}}
\newcommand{\MZ}{\boldsymbol{\mathit{Z}}}


%Tensor 张量
\newcommand{\TSA}{\textsf{\textbf{A}}}
\newcommand{\TSB}{\textsf{\textbf{B}}}
\newcommand{\TSC}{\textsf{\textbf{C}}}
\newcommand{\TSD}{\textsf{\textbf{D}}}
\newcommand{\TSE}{\textsf{\textbf{E}}}
\newcommand{\TSF}{\textsf{\textbf{F}}}
\newcommand{\TSG}{\textsf{\textbf{G}}}
\newcommand{\TSH}{\textsf{\textbf{H}}}
\newcommand{\TSI}{\textsf{\textbf{I}}}
\newcommand{\TSJ}{\textsf{\textbf{J}}}
\newcommand{\TSK}{\textsf{\textbf{K}}}
\newcommand{\TSL}{\textsf{\textbf{L}}}
\newcommand{\TSM}{\textsf{\textbf{M}}}
\newcommand{\TSN}{\textsf{\textbf{N}}}
\newcommand{\TSO}{\textsf{\textbf{O}}}
\newcommand{\TSP}{\textsf{\textbf{P}}}
\newcommand{\TSQ}{\textsf{\textbf{Q}}}
\newcommand{\TSR}{\textsf{\textbf{R}}}
\newcommand{\TSS}{\textsf{\textbf{S}}}
\newcommand{\TST}{\textsf{\textbf{T}}}
\newcommand{\TSU}{\textsf{\textbf{U}}}
\newcommand{\TSV}{\textsf{\textbf{V}}}
\newcommand{\TSW}{\textsf{\textbf{W}}}
\newcommand{\TSX}{\textsf{\textbf{X}}}
\newcommand{\TSY}{\textsf{\textbf{Y}}}
\newcommand{\TSZ}{\textsf{\textbf{Z}}}

% Tensor Element
\newcommand{\TEA}{\textit{\textsf{A}}}
\newcommand{\TEB}{\textit{\textsf{B}}}
\newcommand{\TEC}{\textit{\textsf{C}}}
\newcommand{\TED}{\textit{\textsf{D}}}
\newcommand{\TEE}{\textit{\textsf{E}}}
\newcommand{\TEF}{\textit{\textsf{F}}}
\newcommand{\TEG}{\textit{\textsf{G}}}
\newcommand{\TEH}{\textit{\textsf{H}}}
\newcommand{\TEI}{\textit{\textsf{I}}}
\newcommand{\TEJ}{\textit{\textsf{J}}}
\newcommand{\TEK}{\textit{\textsf{K}}}
\newcommand{\TEL}{\textit{\textsf{L}}}
\newcommand{\TEM}{\textit{\textsf{M}}}
\newcommand{\TEN}{\textit{\textsf{N}}}
\newcommand{\TEO}{\textit{\textsf{O}}}
\newcommand{\TEP}{\textit{\textsf{P}}}
\newcommand{\TEQ}{\textit{\textsf{Q}}}
\newcommand{\TER}{\textit{\textsf{R}}}
\newcommand{\TES}{\textit{\textsf{S}}}
\newcommand{\TET}{\textit{\textsf{T}}}
\newcommand{\TEU}{\textit{\textsf{U}}}
\newcommand{\TEV}{\textit{\textsf{V}}}
\newcommand{\TEW}{\textit{\textsf{W}}}
\newcommand{\TEX}{\textit{\textsf{X}}}
\newcommand{\TEY}{\textit{\textsf{Y}}}
\newcommand{\TEZ}{\textit{\textsf{Z}}}

% Random Scala
\newcommand{\RSa}{\mathrm{a}}
\newcommand{\RSb}{\mathrm{b}}
\newcommand{\RSc}{\mathrm{c}}
\newcommand{\RSd}{\mathrm{d}}
\newcommand{\RSe}{\mathrm{e}}
\newcommand{\RSf}{\mathrm{f}}
\newcommand{\RSg}{\mathrm{g}}
\newcommand{\RSh}{\mathrm{h}}
\newcommand{\RSi}{\mathrm{i}}
\newcommand{\RSj}{\mathrm{j}}
\newcommand{\RSk}{\mathrm{k}}
\newcommand{\RSl}{\mathrm{l}}
\newcommand{\RSm}{\mathrm{m}}
\newcommand{\RSn}{\mathrm{n}}
\newcommand{\RSo}{\mathrm{o}}
\newcommand{\RSp}{\mathrm{p}}
\newcommand{\RSq}{\mathrm{q}}
\newcommand{\RSr}{\mathrm{r}}
\newcommand{\RSs}{\mathrm{s}}
\newcommand{\RSt}{\mathrm{t}}
\newcommand{\RSu}{\mathrm{u}}
\newcommand{\RSv}{\mathrm{v}}
\newcommand{\RSw}{\mathrm{w}}
\newcommand{\RSx}{\mathrm{x}}
\newcommand{\RSy}{\mathrm{y}}
\newcommand{\RSz}{\mathrm{z}}


% Random Vector
\newcommand{\RVa}{\mathbf{a}}
\newcommand{\RVb}{\mathbf{b}}
\newcommand{\RVc}{\mathbf{c}}
\newcommand{\RVd}{\mathbf{d}}
\newcommand{\RVe}{\mathbf{e}}
\newcommand{\RVf}{\mathbf{f}}
\newcommand{\RVg}{\mathbf{g}}
\newcommand{\RVh}{\mathbf{h}}
\newcommand{\RVi}{\mathbf{i}}
\newcommand{\RVj}{\mathbf{j}}
\newcommand{\RVk}{\mathbf{k}}
\newcommand{\RVl}{\mathbf{l}}
\newcommand{\RVm}{\mathbf{m}}
\newcommand{\RVn}{\mathbf{n}}
\newcommand{\RVo}{\mathbf{o}}
\newcommand{\RVp}{\mathbf{p}}
\newcommand{\RVq}{\mathbf{q}}
\newcommand{\RVr}{\mathbf{r}}
\newcommand{\RVs}{\mathbf{s}}
\newcommand{\RVt}{\mathbf{t}}
\newcommand{\RVu}{\mathbf{u}}
\newcommand{\RVv}{\mathbf{v}}
\newcommand{\RVw}{\mathbf{w}}
\newcommand{\RVx}{\mathbf{x}}
\newcommand{\RVy}{\mathbf{y}}
\newcommand{\RVz}{\mathbf{z}}

% Random Matrix
\newcommand{\RMX}{\boldsymbol{\mathrm{X}}}
\newcommand{\RMA}{\boldsymbol{\mathrm{A}}}

\newcommand{\Valpha}{\boldsymbol{\alpha}}
\newcommand{\Vbeta}{\boldsymbol{\beta}}
\newcommand{\Vtheta}{\boldsymbol{\theta}}
\newcommand{\Vlambda}{\boldsymbol{\lambda}}
\newcommand{\Veta}{\boldsymbol{\eta}}
\newcommand{\Vepsilon}{\boldsymbol{\varepsilon}}
\newcommand{\Vmu}{\boldsymbol{\mu}}
\newcommand{\VPhi}{\boldsymbol{\Phi}}
\newcommand{\Vsigma}{\boldsymbol{\sigma}}

\newcommand{\Vrho}{\boldsymbol{\rho}}
\newcommand{\Vgamma}{\boldsymbol{\gamma}}
\newcommand{\Vomega}{\boldsymbol{\omega}}
\newcommand{\Vpsi}{\boldsymbol{\psi}}
\newcommand{\Vzeta}{\boldsymbol{\zeta}}
\newcommand{\Vxi}{\boldsymbol{\xi}}
\newcommand{\Vone}{\boldsymbol{1}}
\newcommand{\Vvarphi}{\boldsymbol{\varphi}}
\newcommand{\MLambda}{\boldsymbol{\Lambda}}
\newcommand{\MSigma}{\boldsymbol{\Sigma}}

\newcommand{\MPi}{\boldsymbol{\mathit{\Pi}}}

%线性变换 书法字体 \CalA
\newcommand{\CalA}{\mathcal{A}}
\newcommand{\CalB}{\mathcal{B}}
\newcommand{\CalC}{\mathcal{C}}
\newcommand{\CalD}{\mathcal{D}}
\newcommand{\CalE}{\mathcal{E}}
\newcommand{\CalF}{\mathcal{F}}
\newcommand{\CalG}{\mathcal{G}}
\newcommand{\CalH}{\mathcal{H}}
\newcommand{\CalI}{\mathcal{I}}
\newcommand{\CalJ}{\mathcal{J}}
\newcommand{\CalK}{\mathcal{K}}
\newcommand{\CalL}{\mathcal{L}}
\newcommand{\CalM}{\mathcal{M}}
\newcommand{\CalN}{\mathcal{N}}
\newcommand{\CalO}{\mathcal{O}}
\newcommand{\CalP}{\mathcal{P}}
\newcommand{\CalQ}{\mathcal{Q}}
\newcommand{\CalR}{\mathcal{R}}
\newcommand{\CalS}{\mathcal{S}}
\newcommand{\CalT}{\mathcal{T}}
\newcommand{\CalU}{\mathcal{U}}
\newcommand{\CalV}{\mathcal{V}}
\newcommand{\CalW}{\mathcal{W}}
\newcommand{\CalX}{\mathcal{X}}
\newcommand{\CalY}{\mathcal{Y}}
\newcommand{\CalZ}{\mathcal{Z}}


% Set
\newcommand{\SetA}{\mathbb{A}}
\newcommand{\SetB}{\mathbb{B}}
\newcommand{\SetC}{\mathbb{C}}
\newcommand{\SetD}{\mathbb{D}}
\newcommand{\SetE}{\mathbb{E}}
\newcommand{\SetF}{\mathbb{F}}
\newcommand{\SetG}{\mathbb{G}}
\newcommand{\SetH}{\mathbb{H}}
\newcommand{\SetI}{\mathbb{I}}
\newcommand{\SetJ}{\mathbb{J}}
\newcommand{\SetK}{\mathbb{K}}
\newcommand{\SetL}{\mathbb{L}}
\newcommand{\SetM}{\mathbb{M}}
\newcommand{\SetN}{\mathbb{N}}
\newcommand{\SetO}{\mathbb{O}}
\newcommand{\SetP}{\mathbb{P}}
\newcommand{\SetQ}{\mathbb{Q}}
\newcommand{\SetR}{\mathbb{R}}
\newcommand{\SetS}{\mathbb{S}}
\newcommand{\SetT}{\mathbb{T}}
\newcommand{\SetU}{\mathbb{U}}
\newcommand{\SetV}{\mathbb{V}}
\newcommand{\SetW}{\mathbb{W}}
\newcommand{\SetX}{\mathbb{X}}
\newcommand{\SetY}{\mathbb{Y}}
\newcommand{\SetZ}{\mathbb{Z}}


%Number Set
\newcommand{\NN}{\mathbb{N}}
\newcommand{\NZ}{\mathbb{Z}}
\newcommand{\NQ}{\mathbb{Q}}
\newcommand{\NR}{\mathbb{R}}
\newcommand{\NC}{\mathbb{C}}
\newcommand{\NK}{\mathbb{K}}

%转置
\renewcommand{\T}{^{\mathrm{T}}}


\newcommand{\Col}[1]{\text{Col}\left(#1\right)}
\newcommand{\Row}[1]{\text{Row}\left(#1\right)}
\newcommand{\Range}[1]{\text{Range}\left(#1\right)}
\newcommand{\Null}[1]{\text{Null}\left(#1\right)}
\newcommand{\nullity}[1]{\text{nullity}\left(#1\right)}
\newcommand{\Ker}[1]{\text{Ker}\left(#1\right)}
\let\olddefinition\definition
\let\oldproposition\proposition
\let\oldtheorem\theorem
\let\oldlemma\lemma
\let\oldaxiom\axiom
\let\oldcorollary\corollary
\let\oldcharacter\character
\let\oldexercise\exercise
\let\oldexam\exam
\let\oldexample\example

\let\oldequation\equation

%\renewcommand{\definition}{\olddefinition\label{dy:\thedefinition}}
%\renewcommand{\proposition}{\oldproposition\label{mt:\theproposition}}
%\renewcommand{\theorem}{\oldtheorem\label{dl:\thetheorem}}
%\renewcommand{\lemma}{\oldlemma\label{yl:\thelemma}}
%\renewcommand{\axiom}{\oldaxiom\label{gl:\theaxiom}}
%\renewcommand{\corollary}{\oldcorollary\label{tl:\thecorollary}}
%\renewcommand{\character}{\oldcharacter\label{xz:\thecharacter}}
%\renewcommand{\exercise}{\oldexercise\label{xt:\theexercise}}
%\renewcommand{\exam}{\oldexam\label{li:\theexam}}
%\renewcommand{\example}{\oldexample\label{cx:\theexample}}

%\renewcommand{\equation}{\oldequation\label{gs:\theequation}}
%\newcommand{\gslbl}[3]{\label{gs:#1:#2:#3}}
%\newcommand{\gsref}[3]{式\ref{gs:#1:#2:#3}}
%\newcommand{\gslast}{式\ref{gs:\theequation}}

%从此处开始看
%标签
\newcommand{\dylbl}[1]{\label{dy:#1}}
\newcommand{\mtlbl}[1]{\label{mt:#1}}
\newcommand{\dllbl}[1]{\label{dl:#1}}
\newcommand{\yllbl}[1]{\label{yl:#1}}
\newcommand{\gllbl}[1]{\label{gl:#1}}
\newcommand{\tllbl}[1]{\label{tl:#1}}
\newcommand{\xzlbl}[1]{\label{xz:#1}}
\newcommand{\xtlbl}[1]{\label{xt:#1}}
\newcommand{\lilbl}[1]{\label{li:#1}}
\newcommand{\cxlbl}[1]{\label{cx:#1}}

%引用
\newcommand{\dyref}[1]{定义\ref{dy:#1}}
\newcommand{\mtref}[1]{命题\ref{mt:#1}}
\newcommand{\dlref}[1]{定理\ref{dl:#1}}
\newcommand{\ylref}[1]{引理\ref{yl:#1}}
\newcommand{\glref}[1]{公理\ref{gl:#1}}
\newcommand{\tlref}[1]{推论\ref{tl:#1}}
\newcommand{\xzref}[1]{性质\ref{xz:#1}}
\newcommand{\xtref}[1]{习题\ref{xt:#1}}
\newcommand{\liref}[1]{例 \ref{li:#1}}
\newcommand{\cxref}[1]{程序\ref{cx:#1}}

%引用最后一个,只能在本节内部,可以不打标签
\newcommand{\dylast}[1][0]{\addtocounter{definition}{-#1}定义\ref{dy:\thedefinition}%
	\addtocounter{definition}{#1}}
\newcommand{\mtlast}[1][0]{\addtocounter{proposition}{-#1}命题\ref{mt:\theproposition}%
	\addtocounter{proposition}{#1}}
\newcommand{\dllast}[1][0]{\addtocounter{theorem}{-#1}定理\ref{dl:\thetheorem}%
	\addtocounter{theorem}{#1}}
\newcommand{\yllast}[1][0]{\addtocounter{lemma}{-#1}引理\ref{yl:\thelemma}%
	\addtocounter{lemma}{#1}}
\newcommand{\gllast}[1][0]{\addtocounter{axiom}{-#1}公理\ref{gl:\theaxiom}%
	\addtocounter{axiom}{#1}}
\newcommand{\tllast}[1][0]{\addtocounter{corollary}{-#1}推论\ref{tl:\thecorollary}%
	\addtocounter{corollary}{#1}}
\newcommand{\xzlast}[1][0]{\addtocounter{character}{-#1}性质\ref{xz:\thecharacter}%
	\addtocounter{character}{#1}}
\newcommand{\xtlast}[1][0]{\addtocounter{exercise}{-#1}习题\ref{xt:\theexercise}%
	\addtocounter{exercise}{#1}}
\newcommand{\lilast}[1][0]{\addtocounter{exam}{-#1}例  \ref{li:\theexam}%
	\addtocounter{exam}{#1}}
\newcommand{\cxlast}[1][0]{\addtocounter{example}{-#1}程序\ref{cx:\theexample}%
	\addtocounter{example}{#1}}

%引用最后一个公式,只能在本章内部,没有超链接,可以不打标签

\newcommand{\gslast}[1][0]{\addtocounter{equation}{-#1}%
	(\theequation)\addtocounter{equation}{#1}}

\begin{document}

\setcounter{chapter}{1}
\chapter*{2021级华东师范大学本科生期末考试}
\section*{《数据科学与工程数学基础》笔试试卷 \qquad 2023.2}
\begin{center}
	学院:数据科学与工程学院 \qquad 考试形式:闭卷 \qquad 所需时间:120分钟
\end{center}
\begin{center}
	考生姓名:\underline{~~~~~~}\underline{~~~~~~}\qquad 学号:\underline{~~~~~~}\underline{~~~~~~}\qquad 专业:\underline{~~~~~~}\underline{~~~~~~}\qquad 任课教师:黄定江
\end{center}
{ \begin{center}
		\begin{tabular}{|c|c|c|c|c|c|c|c|c|c|}
			\hline 
			题目 & 一 & 二 & 三 & 四 & 五 & 六 & 七 & 八&总分 \\ 
			\hline 
			满分 & 13 & 12 & 12 & 13 & 13 & 12 & 13 & 12 & 100  \\ 
			\hline 
			得分 &  &  &  &  &  &  &  & &  \\ 
			\hline 
			阅卷人 &  &  &  &  &  &  &  & &  \\ 
			\hline 
		\end{tabular} 
		
\end{center}}

\begin{exercise}(13')
	假设 $Q \in \mathbb{R}^{n \times n} \backslash\{0\}$ 是一个投影矩阵.\\
	\begin{itemize}
		\item [(1)] 证明:$Q z=z, \ \forall z \in \mathcal{R}(Q)$以及$Q y-y \in \mathcal{N}(Q), \ \forall y \in \mathbb{R}^n$.
		\item [(2)] 证明:$Q$的特征值$\lambda \in \Lambda(Q) \subseteq\{0,1\}$。
		\item [(3)]  假设$\mathcal{R}(Q)=\operatorname{span}\left(u_1, \ldots, u_r\right)$ , $\mathcal{N}(Q)=\operatorname{span}\left(v_{r+1}, \ldots, v_n\right)$ 这里$r$表示投影矩阵的秩,请据此写出$Q$的特征分解 $Q=X D X^{-1}$的具体形式。
		\item [(4)] 证明:当 $Q \neq I_n$, $\operatorname{det}(Q)=0$。
	\end{itemize}
	
\end{exercise}
\begin{Solution}
(1)
$\forall \ y \in \mathcal{R}(Q)$必存在$x \in \mathbb{R}^n$,使得$y=Q x$ 成立。因此,$Q y=Q^2 x=Q x=y$。另外,
$\forall x \in \mathbb{R}^n$,$Q(Q x-x)=Q^2 x-Q x=Q x-Q x=0$。
\hfill \textcolor{red}{\textbf{(3分)}}

(2)
对 $\lambda \in \Lambda(Q)$,$x \in \mathbb{R}^n \backslash\{0\}$,有 $Q x=\lambda x$. 由于 $Q=Q^2$,  $\lambda x=Q x=Q(Q x)=Q(\lambda x)=\lambda Q x=\lambda^2 x$。因为 $x \neq 0 \in \mathbb{R}^n$,故 $\lambda=\lambda^2$,$\lambda \in\{0, 1\}$。因此 $\Lambda(Q) \subseteq\{0,1\}$.	\hfill \textcolor{red}{\textbf{(3分)}}

(3)
由(1)可知,$\forall i = 1,...,r, u_i\in \mathcal{R}(Q), Q u_i=u_i$,以及$\forall j = r+1,...,n,v_j\in \mathcal{N}(Q),Q v_j=0$。
故令 $$X:=\left(u_1|\cdots| u_r\left|v_{r+1}\right| \cdots \mid v_n\right) \in \mathbb{R}^{n \times n},$$ $$D:=\operatorname{diag}_{n \times n}(\underbrace{1, \ldots, 1}_{r \text { times }}, 0 \ldots, 0) \in \mathbb{R}^{n \times n},$$
此时$Q=X D X^{-1}$。
\hfill \textcolor{red}{\textbf{(4分)}}

(3) 反证$\operatorname{det}(Q) \neq 0 \Longrightarrow Q=I_n$。由于 $\operatorname{det}(Q) \neq 0$,故$Q$ 可逆。故由 $Q^2=Q$,得 $Q^{-1} Q^2=Q^{-1} Q$,即$Q=I_n$。 \hfill \textcolor{red}{\textbf{(3分)}}
	
\end{Solution}

\begin{exercise}(12')
	已知矩阵\[ \MM = \begin{pmatrix}
		0&-3&3\\0&0&4\\2&1&1
	\end{pmatrix}
	\]
	
	\begin{itemize}
		\item [(1)] 先求矩阵$\MM$的QR分解,然后根据QR分解求解方程组$\MM \Vx = \Vb$,其中$\Vb=[6, 4, 1]^\top$;
		\item [(2)] 请根据你对LU分解、Cholesky分解和QR分解的理解,谈谈它们之间的联系与差异。
	\end{itemize}
	
\end{exercise}
\begin{Solution}
	(1)因为 $\boldsymbol{\alpha}_1=(0,0,2)^T$, 记 $a_1=\left\|\alpha_1\right\|_2=2$, 令 $\quad \boldsymbol{w}_1=\frac{\alpha_1-a_1 e_1}{\left\|\alpha_1-a_1 e_1\right\|_2}=$ $\frac{1}{\sqrt{2}}(-1,0,1)^T$, 则
	$$
	\boldsymbol{H}_1=I-2 \boldsymbol{w}_1 \boldsymbol{w}_1^H=\left(\begin{array}{ccc}
		0 & 0 & 1 \\
		0 & 1 & 0 \\
		1 & 0 & 0
	\end{array}\right)
	$$
	从而
	$$
	\boldsymbol{H}_1 \boldsymbol{M}=\left(\begin{array}{ccc}
		2 & 1 & 1 \\
		0 & 0 & 4 \\
		0 & -3 & 3
	\end{array}\right)
	$$
	记 $\boldsymbol{\beta}=(0,-3)^T$, 则 $b_2=\left\|\boldsymbol{\beta}_2\right\|_2=3$, 令 $\boldsymbol{w}_2=\frac{\boldsymbol{\beta}_2-b_2 e_1}{\left\|\boldsymbol{\beta}_2-b_2 \boldsymbol{e}_1\right\|_2}=\frac{1}{\sqrt{2}}(-1,-1)^T$
	$$
	\widetilde{\boldsymbol{H}}_2=\boldsymbol{I}-2 \boldsymbol{w}_2 \boldsymbol{w}_2^{\boldsymbol{H}}=\left(\begin{array}{cc}
		0 & -1 \\
		-1 & 0
	\end{array}\right)
	$$
	记
	$$
	\boldsymbol{H}_2=\left(\begin{array}{cc}
		1 & 0^T \\
		0 & \widetilde{\boldsymbol{H}}_2
	\end{array}\right)=\left(\begin{array}{ccc}
		1 & 0 & 0 \\
		0 & 0 & -1 \\
		0 & -1 & 0
	\end{array}\right)
	$$
	则
	$$
	\boldsymbol{H}_2\left(\boldsymbol{H}_1 \boldsymbol{M}\right)=\left(\begin{array}{ccc}
		2 & 1 & 1 \\
		0 & 3 & -3 \\
		0 & 0 & -4
	\end{array}\right)=\boldsymbol{R}
	$$
	易知 $\mathrm{Rx}=H_2 H_1 b=(1,-6,-4)^T$, 故可求得
	$
	\mathrm{x}=(1 / 2, \quad-1, \quad 1)^T
	$ \hfill \textcolor{red}{\textbf{(7分)}}
	
	(2)(总体上,只要理解正确即可)这三种分解方式的联系在于将一个非三角矩阵转化为三角矩阵,这样便可以有效提高实际应用中一系列方程组求解的效率。它们区别在于LU分解和Cholesky分解是基于高斯消去法,分别适用于一般地可逆矩阵和对称矩阵,QR分解则是利用正交投影的方法,将向量分别投影到左边平面上的方式,从而达到对角化。 \hfill \textcolor{red}{\textbf{(5分)}}
\end{Solution}


\begin{exercise}(12')
	完成下列矩阵函数求梯度:
	\begin{itemize}
		\item [(1)] 求行列式函数$f(\MX)=|\MX^3|$的梯度矩阵,其中变量$\MX \in \NR^{n\times n} $非奇异;
		\item [(2)] 求迹函数$f(\MX) = Tr(\MX^\top\MX^{-1}\MA)$的梯度矩阵,其中变量$\MX \in \NR^{n\times n} $非奇异。
	\end{itemize}
\end{exercise}
\begin{Solution}
	(1)因为$d|\MX| =Tr(|\MX|\MX^{-1}d\MX)$,因此
	$$
	\begin{aligned}
		d(|\MX^{3}|)&= Tr( \MX \MX^{-3} d\MX^{3})\\
		&=Tr(|\MX^3| X^{-3} (d\MX \cdot \MX^2+\MX d\MX \cdot \MX + \MX^2d\MX))\\
		&=Tr(3|\MX|^3 \MX^{-1} d\MX)
	\end{aligned}
	$$
	所以梯度矩阵$3|\MX|^3 (\MX^{-1})^\top$。
	\hfill \textcolor{red}{\textbf{(6分)}}
	
	(2)
	
	$$
	\begin{aligned}
		dTr(\MX^\top \MX^{-1} \MA) &= Tr(d(\MX^\top \MX^{-1} \MA))\\
		&= Tr( d\MX^\top \MX^{-1}\MA -\MX^\top \MX^{-1} d\MX \MX^{-1} \MA)\\
		&= Tr((\MA^\top (\MX^{-1})^\top -\MX^{-1}\MA\MX^\top \MX^{-1}) d\MX)
	\end{aligned}
	$$
	
	即
	
	$$
	\begin{aligned}
		\frac{\partial Tr(\MX^\top \MX^{-1} \MA)}{\partial \MX} = (\MA^\top (\MX^{-1})^\top -\MX^{-1}\MA\MX^\top \MX^{-1})^\top
	\end{aligned}
	$$\hfill \textcolor{red}{\textbf{(6分)}}
	
\end{Solution}


\begin{exercise}(13')
	矩阵\[ \MA = \begin{pmatrix}
		7/5&-1/5\\
		1/5&7/5\\
		0&0
	\end{pmatrix}
 \]
 \begin{itemize}
 	\item [(1)] 计算矩阵$ \MA $的SVD分解;
 	\item [(2)] 假设$\MM$是任意一个非奇异$n \times d$的矩阵,已知其奇异值分解为$\MM = \MU \Sigma \MV^\top$,其中$\MU = [u_1, u_2, \cdots, u_n]$,$\Sigma = \diag(\lambda_1, \lambda_2, \cdots, \lambda_n)$,$\MV = [v_1, v_2, \cdots, v_n]$。请分别写出矩阵$\MM$的逆矩阵的SVD分解;
 	\item [(3)] 请写出(2)中$\MM$的转置矩阵的SVD分解。
 \end{itemize}
\end{exercise}
\begin{Solution}
	(1)易计算
			$$\MA^\top \MA =\begin{pmatrix}
				2& 0\\
				0& 2\end{pmatrix}$$
		求得其特征值为$2,9$。对应的特征向量分别为$(1, 0)^\top, (0, 1)^\top$,因此,
		$$\MV =\begin{pmatrix}
			1& 0\\
			0& 1\end{pmatrix}$$\hfill \textcolor{red}{\textbf{(3分)}}
		
		由$\MA \MV$与$\MU$的关系可直接计算出
		$$\MU =\begin{pmatrix}
			7/5\sqrt{2}& -1/5\sqrt{2}& 0\\
			1/5\sqrt{2}& 7/5\sqrt{2} & 0\\
			0& 0 & 1
	\end{pmatrix}$$\hfill \textcolor{red}{\textbf{(3分)}}

	(2)因为$\MM$的SVD分解为
	$$
	\MM=U \Sigma V^{\mathrm{T}}=\left(u_1|\cdots| u_n\right)\left[\operatorname{diag}_{n \times n}\left(\lambda_1, \ldots, \lambda_n\right)\right]\left(v_1|\cdots| v_n\right)^{\mathrm{T}}
	$$
	其中 $U, V \in \mathbb{R}^{n \times n}$ 正交, $\lambda_1 \geq \lambda_2 \geq \cdots \geq \lambda_n \geq 0$。 因为$\MM$可逆,有 $\operatorname{rank}(\MM)=n$ , 故$\lambda_i>0$ , $\forall i \in\{1, \ldots, n\}$.又由 $\MM=U \Sigma V^{\mathrm{T}}$ 有 $\MM v_i=\lambda_i u_i$ $\forall i \in\{1, \ldots, n\}$ ,因此 
	$$\MM^{-1} u_i=\MM^{-1}\left(\frac{1}{\lambda_i} \MM v_i\right)=\frac{1}{\lambda_i} v_i$$
	其中 $\frac{1}{\lambda_n} \geq \cdots \geq \frac{1}{\lambda_2} \geq \frac{1}{\lambda_1}>0$。综上
	$$
	\MM^{-1}=\left(v_n|\cdots| v_2 \mid v_1\right)\left[\operatorname{diag}_{n \times n}\left(\frac{1}{\lambda_n}, \ldots, \frac{1}{\lambda_2}, \frac{1}{\lambda_1}\right)\right]\left(u_n|\cdots| u_2 \mid u_1\right)^{\mathrm{T}},
	$$
	整理可得
	$$
	\MM^{-1}=(V P)\left(P \Sigma^{-1} P\right)(U P)^{\mathrm{T}}.
	$$
	记 $P:=\left(e_n|\cdots| e_2 \mid e_1\right) \in \mathbb{R}^{n \times n}$. (由于 $P P^{\mathrm{T}}=P^2=I_n$, $P$是正交阵, 且$V P$, $U P$ 是正交阵)。\hfill \textcolor{red}{\textbf{(5分)}}
	
	(3)因为$\MM$的SVD分解为
	$$
	\MM=U \Sigma V^{\mathrm{T}}
	$$
	其中 $U, V \in \mathbb{R}^{n \times n}$ 正交。 
	易得
	$$
	\MM^\top=V \Sigma U^T.
	$$
	显然$V$, $U$ 是正交阵。\hfill \textcolor{red}{\textbf{(2分)}}
\end{Solution}


\begin{exercise}(13')
	已知矩阵\[ \MM = \begin{pmatrix}
		2&0\\
		-1&2\\
		0&3\\
	\end{pmatrix}
	\]
	\begin{itemize}
		\item [(1)] 计算矩阵 $\MM$ 的 $1$ 范数和$\infty$范数;
		\item [(2)] 计算矩阵 $\MM$ 的 $F$ 范数和$2$范数。
		\item [(3)] 试比较任意矩阵$\MA$的 $F$ 范数和$2$范数的大小,并给出理由。
	\end{itemize}
\end{exercise}
\begin{Solution}
	(1)	
	$$\|\MM\|_1=\max\{2+\|-1\|,0+2+3\}=5$$ 
	$$\|\MM\|_{\infty}=\max\{2+\|0\|,\|-1\|+2, 0+3\}=3$$
	\hfill \textcolor{red}{\textbf{(4分)}}
	
	(2)$\MM^\top \MM=\begin{pmatrix}5&-2\\-2&13\end{pmatrix}$
	
	$\lambda_{max}(\MM^\top \MM)=9+2\sqrt{2}$
	
	$\|\MM\|_2=\sqrt{9+2\sqrt{2}}$ 
	
	易求得
	$\|\MM\|_F=\sqrt{18}=3\sqrt{2}$ 
	\hfill \textcolor{red}{\textbf{(5分)}}
	
	(3)显然$F$范数大于$2$范数,因为$F$范数的平方是所有奇异值的平方之和,而$2$范数的平方是最大奇异值的平方,因此可知这一点成立。 \hfill \textcolor{red}{\textbf{(4分)}}
\end{Solution}


\begin{exercise}(12')
	求解以下问题:
\begin{itemize}
	\item [(1)] 证明: Gauss概率密度函数的累积分布函数
	\begin{displaymath}
		\Phi(x)=\frac{1}{\sqrt{2\pi}}\int^{x}_{-\infty}e^{-u^{2}/2}du
	\end{displaymath}
	是对数-凹函数. 即$ \log(\Phi(x)) $是凹函数。
	\item [(2)] 设总体 $X \sim E(\theta), X_{1}, X_{2}, \ldots, X_{n}$ 为来自总体 $X$ 的样本, $\theta$ 的先验分布
	为指数分布 $E(\lambda)$ ( $\lambda$ 已知), 求 $\theta$ 的最大后验估计。
\end{itemize}

\end{exercise}
\begin{Solution}
	(1)由题意得,
	$$\Phi(x)=\frac{1}{\sqrt{2\pi}}\int^{x}_{-\infty}e^{-u^{2}/2}du $$
	$$\Phi^{'}(x)=\frac{1}{\sqrt{2\pi}}e^{-x^{2}/2}$$
	$$\Phi^{''}(x)=\frac{1}{\sqrt{2\pi}}e^{-x^{2}/2}(-x)$$
	$$(\Phi^{'}(x))^{2}=\frac{1}{2\pi}e^{-x^{2}}$$
	$$\Phi(x)\log\Phi^{''}(x) = \frac{1}{2\pi}\int^{x}_{-\infty}e^{-u^{2}/2}du \cdot e^{-x^{2}/2}(-x)$$
	\hfill \textcolor{red}{\textbf{(3分)}}
	
	当$x\geq 0$时, $(\Phi^{'}(x))^{2} \geq 0 \geq \Phi(x)\Phi^{''}(x)$. \\
	当$x < 0$时, 由于$\frac{u^{2}}{2}$是凸函数, 则
	$$\frac{u^{2}}{2} \geq \frac{x^{2}}{2} + (u-x)x\geq xu - \frac{x^{2}}{2}$$
	所以,
	$$\int^{x}_{-\infty}e^{-u^{2}/2}du \leq  \int^{x}_{-\infty}e^{\frac{x^{2}}{2}- xu}du$$
	$$=  e^{\frac{x^{2}}{2}\cdot\frac{e^{-xu}}{-x}\big|_{u=-\infty}^{x}}$$
	$$=  e^{\frac{x^{2}}{2}}\cdot\frac{e^{-x^{2}}}{-x}$$
	因此$\Phi(x)\Phi^{''}(x) \leq \frac{1}{2\pi}e^{-x^{2}} = (\Phi^{'}(x))^{2}$, $\Phi(x)$是对数凹函数.
	\hfill \textcolor{red}{\textbf{(3分)}}
	
	(2)因为先验概率密度函数为:
	$$
	\pi(\theta)=\left\{\begin{array}{cc}
		\lambda e^{-\lambda \theta} & \theta>0 \\
		0 & \theta \leq 0
	\end{array}\right.
	$$
	样本 $\left(X_{1}, X_{2}, \ldots, X_{n}\right)$ 的联合概率密度为:
	$$
	q(\mathbf{x} \mid \theta)=\prod_{n=2}^{n} f\left(x_{i} \mid \theta\right)=\left\{\begin{array}{cc}
		\theta^{n} e^{-\theta \sum_{i=1}^{n} x_{i}} & x_{1}, x_{2}, \ldots, x_{n}>0 \\
		0 & \text { 其它 }
	\end{array}\right.
	$$
	所以 $\theta$ 的后验分布密度
	$$
	\begin{gathered}
		h(\theta \mid \mathbf{x}) \propto \theta^{n} e^{-\left(\lambda+\sum_{i=1}^{n} x_{i}\right) \theta} \\
		\ln h(\theta \mid \mathbf{x})=n \ln \theta-\left(\lambda+\sum_{i=1}^{n} x_{i}\right) \theta+\ln c(\mathbf{x}) \\
		\frac{\partial \ln h(\theta \mid \mathbf{x})}{\partial \theta}=\frac{n}{\theta}-\left(\lambda+\sum_{i=1}^{n} x_{i}\right)=0
	\end{gathered}
	$$
	求得 $\theta$ 的最大后验估计为 $\hat{\theta}=\frac{1}{\bar{x}+\lambda / n}$ 。当 $n \rightarrow \infty$ 时, $\hat{\theta} \rightarrow \frac{1}{\bar{x}}$, 与传统意义下的极 大似然估计是一致的。\hfill \textcolor{red}{\textbf{(6分)}}
\end{Solution}




\begin{exercise}(13')
	求解下述问题:
	\begin{itemize}
		\item[(1)] 写出下述非线性规划的KKT条件,并求解
		\begin{align*}
			\quad \texttt{maximize} \quad & f(x)=(x-4)^2 \\
			\texttt{suject to} \quad & 1\leq x \leq 5 
		\end{align*}
		\item [(2)] 用Lagrange乘子法证明:矩阵$ \MA\in\NR^{m\times n} $的2范数 \[ \Vert\MA\Vert_{2}=\max_{\Vert\Vx\Vert_2=1,\Vx\in\NR^n}\Vert\MA\Vx\Vert_{2} \] 的平方是$ \MA\T\MA $的最大特征值。
	\end{itemize}
\end{exercise}
\begin{Solution}
	(1)原问题等价于
	\begin{equation*}
		\left\{
		\begin{aligned}
			& \texttt{minimize}  \quad - f(x)=(x-4)^2\\
			& g_{1}(x) = -x + 1 \leq 5\\
			& g_{2}(x) = x - 5 \leq 0 \\
		\end{aligned}
		\right.
	\end{equation*}
	求目标函数和约束函数的梯度得,
	$$\nabla_{x}f(x) = -2(x-4), \nabla_{x}g_{1}(x) = -1, \nabla_{x}g_{2}(x) = 1$$
	将约束引入广义Lagrange乘子$v_{1}, v_{2}$, 在KKT条件上有
	\begin{equation*}
		\left\{
		\begin{aligned}
			& -2(x^{*} - 4) - v_{1}^{*} + v_{2}^{*} = 0\\
			& v_{1}^{*}(-x^{*} + 1) = 0\\
			& v_{2}^{*}(x^{*} - 5) = 0 \\
			& v_{1}^{*} \geq 0, v_{2}^{*}\geq 0
		\end{aligned}
		\right.
	\end{equation*}
\hfill \textcolor{red}{\textbf{(4分)}}

	若$v_{1}^{*}\neq 0, v_{2}^{*}\neq 0$, 无解. \\
	若$v_{1}^{*} = 0, v_{2}^{*}\neq 0$, 得$x^{*} = 5, v_{2}^{*} = 2, -f(x^{*}) = -1$. \\
	若$v_{1}^{*} \neq 0, v_{2}^{*}= 0$, 得$x^{*} = 1, v_{1}^{*} = 6, -f(x^{*}) = -9$. \\
	若$v_{1}^{*} = 0, v_{2}^{*}= 0$, 得$x^{*} = 4, f(x^{*}) = 0$. \\
	因此最优点$x^{*} = 1$, $\texttt{maximize}f(x) = 9$。
	\hfill \textcolor{red}{\textbf{(3分)}}
	
	(2)优化问题为\[ 	\begin{aligned}
		\texttt{maximize} \quad & f(\Vx)=\Vx\T\MA\T\MA\Vx \\
		\texttt{suject to} \quad & \Vx\T\Vx=1
	\end{aligned} \]
	Lagrange函数为:
	\[ L(\Vx)=\Vx\T\MA\T\MA\Vx-\lambda(\Vx\T\Vx-1) \]
	\[ \frac{\partial L}{\partial \Vx}= 2\MA\T\MA\Vx -2\lambda\Vx\]
	令$ \frac{\partial L}{\partial \Vx}=0 $,有:
	\[ \MA\T\MA\Vx =\lambda \]
	\hfill \textcolor{red}{\textbf{(3分)}}
	
	这表示在$ f(\Vx) $的极大值点,$ \Vx $是$ \MA\T\MA $的特征向量,$ \lambda $是对应的特征值。此时,\[  f(\Vx)=\Vx\T\MA\T\MA\Vx=\Vx\T\lambda\Vx=\lambda\Vx\T\Vx=\lambda  \]
	因此说明,为使$ f(\Vx) $最大,$ f(\Vx)=\lambda_{max}(\MA\T\MA) $,其中$ \lambda_{max} $表示最大特征值。即
	\[ \Vert\MA\Vert_{2}^2=\lambda_{max}(\MA\T\MA) \]
	\hfill \textcolor{red}{\textbf{(3分)}}
\end{Solution}

\begin{exercise}(12')
	求解如下优化问题
	\begin{itemize}
		\item [(1)] 考虑问题
		$$
		\min f(\Vx)=3 x_1^2+3 x_2^2-2x_1^2 x_2 .
		$$
		从初始点$\Vx^{(0)}=(0,1)^{\mathrm{T}}$出发, 写出用 最速下降法迭代两步的求解过程,并说明迭代是否可以终止。
		\item[(2)] 尽管牛顿法在求解无约束优化问题时,具有更高阶的收敛速度,但在处理大规模问题时,该方法涉及的Hessian矩阵及其逆矩阵的计算是非常耗时的。因此,研究者们探究了对Hessian矩阵逆近似的迭代方法,也就是拟牛顿法或变尺度法。通常这样近似Hessian矩阵或逆矩阵需要保持原有的性质,即满足割线方程:
		$$\Delta x^{(k)} = \bar{H}^{(k+1)} \Delta g^{(k)},$$
		其中$\bar{H}^{(k)}$表示矩阵的Hessian矩阵逆的近似。如果已经得到上一轮迭代时的近似$\bar{H}^{(k)}$,则一般地做法是通过秩一修正或秩二修正得到下一轮的近似矩阵$\bar{H}^{(k+1)}$。当采用秩二修正时,便得到著名的DFP法。现已知第$k$次迭代时的$\Delta x^{(k)}, \bar{H}^{(k)}, \Delta g^{(k)}$,请给出下一轮迭代的Hessian矩阵逆的近似的推导过程。
	\end{itemize}
\end{exercise}

\begin{Solution}
	(1)由题意得, $\nabla_{\Vx}f(\Vx) = (6x_1-4x_1x_2, 6x_2-2x_1^2)^T$,故$g^{(0)}=\nabla_{\Vx}f(\Vx^{(0)})=(0, 6)$。现设最速下降法的步长为$\lambda$, 那么
	\begin{align*}
		f(\Vx^{(0)} - \lambda g^{(0)}) = 3(1-6\lambda^2)
	\end{align*}
	在$\nabla_{\Vx} f(\Vx^{(0)})$方向上, 使$f(\Vx)$最小的$\lambda$为
	$$\lambda = \frac{1}{6}$$
	所以,
	$$\Vx^{(1)} = \Vx^{(0)} - \frac{1}{2}\nabla_{\Vx}f(\Vx^{(0)}) = (0, 0)^{T}$$
	易知此时$\Vx^{(1)}$处的梯度已经为0,可见已经收敛。显然
	$$\Vx^{(2)} = (0, 0)^{T}$$
	\hfill \textcolor{red}{\textbf{(7分)}}
	
	(2)可设
	\begin{equation*}
		\overline{\MH}^{(k+1)}=\overline{\MH}^{(k)}+a\Vu\Vu^T+b\Vv\Vv^T
	\end{equation*}
	其中$\Vu,\Vv\in\NR^n,~a,b\in\NR$待定。
	
	同上所述,仍然利用割线方程,有
	\begin{equation*}
		\begin{aligned}
			\Delta\Vx^{(k)}
			& =\overline{\MH}^{(k+1)} \Delta\Vg^{(k)} \\
			& = (\overline{\MH}^{(k)}+a\Vu\Vu^T+b\Vv\Vv^T) \Delta\Vg^{(k)} \\
		\end{aligned}
	\end{equation*}
	整理得
	\[ a(\Vu^T \Delta\Vg^{(k)})\Vu + b(\Vv^T \Delta\Vg^{(k)})\Vv = \Delta\Vx^{(k)} - \overline{\MH}^{(k)} \Delta\Vg^{(k)}. \]
	因此,$\Vu,\Vv$的线性组合等于$\Delta\Vx^{(k)} - \overline{\MH}^{(k)} \Delta\Vg^{(k)}$。同样地,不妨令$\Vu = \Delta\Vx^{(k)}$,$\Vv = \overline{\MH}^{(k)} \Delta\Vg^{(k)}$,则代入可得
	\[ a = \frac{1}{(\Delta\Vx^{(k)})^T \Delta\Vg^{(k)}}, \]
	\[ b = -\frac{1}{(\Delta\Vg^{(k)})^T \overline{\MH}^{(k)} \Delta\Vg^{(k)}}. \]
	从而,得到更新公式:
	\begin{equation*}\label{eq_dfp}
		\overline{\MH}^{(k+1)}=\overline{\MH}^{(k)} + \frac{\Delta\Vx^{(k)}) \Delta\Vx^{(k)})^T}{(\Delta\Vx^{(k)})^T \Delta\Vg^{(k)}} - \frac{\overline{\MH}^{(k)} \Delta\Vg^{(k)} (\overline{\MH}^{(k)} \Delta\Vg^{(k)})^T}{(\Delta\Vg^{(k)})^T \overline{\MH}^{(k)} \Delta\Vg^{(k)}} .
	\end{equation*}
	\hfill \textcolor{red}{\textbf{(5分)}}
\end{Solution}




\end{document}

